\section{Introduction}

\subsection{Les moteurs physiques}

Invisibles, les phénomènes physiques qui régissent le fonctionnement de notre univers sont pourtant omniprésents et universels. Etudiées depuis des siècles, les lois les décrivant ont été à maintes reprises redéfinies et affinnées et il est de nos jours indispensable de pouvoir les modéliser de façon fidèle ou tout du moins d'être capable de les approximer de façon plausible. 
Une simulation industrielle visant par exemple à reproduire virtuellement les interactions entre les différentes pièces qui composent une automobile doit être capable de reproduire de façon réaliste la friction des pneumatiques sur le sol, l'influence de la gravité sur le véhicule, le comportement thermique et volumique des fluides qu'il contient ainsi que de nombreux autres aspects de son fonctionnement. Le système simulant tous ces facteurs est appellé moteur physique. Dans cet exemple, les enjeux de sécurité et de qualité sont grands et des résultats d'une précision extrême sont exigés. Les calculs à mettre en jeu pour les obtenir peuvent donc se permettre d'être très coûteux et de se baser sur des modélisations mathématiques complexes. Il n'est pas rare que ces processus de simulation soient réparties sur plusieurs machines et s'étalent sur une durée de calcul de plusieurs heures.

A contrario, les moteurs physiques d'autres types de système complexe ne peuvent se permettre une telle latence et doivent fonctionner en temps réel. La contrainte est encore plus forte lorsque lorsque le moteur physique doit cohabiter avec d'autres modules gérant différents aspects de l'application. On pense notamment aux jeux vidéo, qui partagent le temp de calcul alloué entre le moteur graphique, le moteur d'intelligence artificielle, la gestion du son, la gestion du réseau, et bien sûr le moteur physique. Dans le cadre de ce type d'application, auquel on peut greffer la réalité virtuelle et ses applications dérivées, la contrainte la plus importante est le temps d'éxecution et non la précision des résultats. On ne cherche plus à obtenir des données exactes mais une représentation plausible du réel. Ainsi, certains raccourcis pourront être tolérés et une part de réalisme est sacrifiée au profit de la vitesse.

La physique est un champ très vaste dont les disciplines vont de l'acoustique à l'électronique. Néanmoins, le spectre d'action des moteurs physique se limite à la mécanique classique, dite mécanique newtonienne. Ce sous-domaine répond à des questions telles que : Comment réagira cette balle si on la lance sur un mur ? Quelle est l'influence d'une planète sur un objet spatial donné ? Pourquoi un liquide visqueux dispose-t-il d'une faible vitesse d'écoulement ? Quelles déformations engendrera un choc entre deux voitures ?

La mécanique newtonienne est elle-même un champ très large et la précision d'une modélisation ne suffit pas à la différenciation entre tous les moteurs physiques. Souvent, un moteur physique sera spécialisé. Certains simuleront les interactions entre corps rigides comme une boîte en plastique tombant sur le sol. Certains simuleront le comportement d'objets déformables comme une balle de caoutchouc jeté sur un mur. D'autres se concentreront sur les réactions entre liquides ou entre gaz.

\subsection{Travail à accomplir}

L'objectif de ce projet est de concevoir un moteur physique de base permettant de gérer les interactions entre corps rigides convexes. On parle de corps rigides dans la mesure o\`u les objets mis en jeu seront indéformables et incassables. Dans un tel contexte, une tasse en porcelaine surmontée d'un bloc de granite d'une tonne ne serait aucunement endommagée. On parle de corps convexes car se limiter dans un premier temps à ce type de structure autorise certaines facilités dans les calculs. Une piste pour étendre la simulation aux solides concaves sera présentée dans la dernière partie.

La contrainte principale de notre application sera le temps. On veut concevoir une simulation éxecutable en temps réel de telle façon que si l'utilisateur modifie l'environnement de la simulation à un instant quelconque, une réaction à cette interaction soit immédiatement générée. Certains raccourcis dans les calculs ainsi que plusieurs approximations seront acceptés. Le fonctionnement du moteur se base néanmoins sur des lois biens connues de la mécanique de Newton et ses résultats ne devront pas s'éloigner de façon demesurée de ceux que l'on retrouverait dans une situation réelle.

\subsection{\'Etude de cas}

La simulation du comportement physique d'un corps se divise en plusieurs phases. Afin de les détailler, analysons une situation concrète. Si l'on tient une balle dans notre main et que nous la lachons au dessus d'un plan faiblement incliné, quelles seraient les étapes que cet objet traversera avant d'arriver à un état de repos (id est, un état d'immobilité totale) ?

\subsubsection{La chute}

La main s'ouvre et laisse s'échapper la balle. Notre appréhension du monde qui nous entoure nous permet de prévoir que l'objet tombera et accélérera vers le bas. Ce phénomène est quantifié par la seconde loi de Newton.

\[
\vec{a} = \frac{1}{m} \sum{F_i}
\]

Cette règle décrit l'accélération $\vec{a}$ d'un corps comme étant le produit de l'inverse de sa masse $m$ et de la somme des forces qui lui sont appliquées. Dans notre exemple, on lache la balle sans lui donner d'élan initial et la seule influence qu'elle subit au cours de sa chute est celle de la gravité. La gravité terrestre est une force de $9.81 N$ dirigée vers le noyau de la planète mais dans la simulation, on peut la réduire à une force dirigée vers le "bas" (dans la direction négative de l'axe $y$). Afin de bénéficier d'un moteur physique versatile, la puissance de la gravité pourra être modifiée, pour simuler une situation lunaire par exemple, ou être totalement annulée, pour simuler une situation en apesanteur. En réalité, la gravité ne sera pas codée "en dur" mais appartiendra à une liste de forces environnementales qui contiendra toutes les influences que le monde extérieur applique aux objets qu'il contient. On pourra par exemple modéliser, entre autres, la résistance de l'air ou le roulis irrégulier du vent.

Cette observation nous oriente sur la façon dont l'on pourra modéliser les déplacements d'objets soumis à des forces extérieures. \`A Chaque corps seront associées des quantités physiques telles que la position, la vitesse et l'accélération. C'est de la façon dont évolueront ces variables que dépend le réalisme du moteur physique. Dans la première partie de ce compte-rendu, on précisera donc les méthodes employés pour simuler la dynamique des corps.

\subsubsection{Le rebond}

Alors que la balle s'approche du plan incliné, on s'attend naturellement à ce qu'elle entre en contact avec ce dernier et qu'une réaction proportionnelle à la puissance du choc soit produite. Cette réaction dépend de nombreux facteurs, notamment des masses des objets et de leurs vitesses respectives.

Le moteur physique devra être capable de générer une réaction réaliste en fonction de certaines variables de la simulation. La détermination de cette réponse physique passe par une formule présentée dans la seconde partie. Néanmoins, le travail le plus complexe n'est pas de calculer une réponse mais de détecter une collision. Plusieurs processus géométriques devront être mis en place afin de vérifier si une collision a lieu et si tel est le cas, afin de mesurer quels points précis des deux corps entrent en contact.

Une difficulté supplémentaire vient du fait que la simulation est mise à jour de façon discrète. Lorsqu'une collision sera détectée entre deux corps, il est presque impossible de se retrouver dans une situation de contact parfait. On aura plutôt des contacts pénétrants au sein desquels l'intégrité physique des corps est corrompue et le objets rentrent l'un dans l'autre. Une des tâches du moteur physique sera de pallier à ce problème en recalant les corps dans la position de contact parfait qu'ils auraient dû atteindre.

Cette phase de la vie d'un corps rigide est la plus courte, puisqu'instantanée, mais demandera paradoxalement le plus de travail. Les considérations géométrique qui entrent en jeu, ainsi que les limites à contourner, imposées par l'arithmétique des ordinateurs, seront détaillées dans la seconde section.

\subsubsection{Le repos}

Les rebonds sur le plan sont de plus en plus faible, jusqu'à ce que la balle n'ait plus assez d'énergie kinétique pour s'élever à nouveau. Puisque qu'elle repose sur un plan légerement incliné, elle roule quelques instants, de plus en plus lentement, jusqu'à parvenir à un arrêt total.

Son état de repos laisse penser que plus aucune force ne s'applique sur elle. Contrairement aux apparences, dans le monde réel, ce n'est pas parce qu'un corps est fixe qu'il est libre de toute influence. La gravité n'a pas disparu et pourtant l'accélération de la balle est nulle, puisqu'elle est immobile. La troisième loi de Newton est là pour démystifier cette situation :

\begin{quote}
Tout corps $A$ exerçant une force sur un corps $B$ subit une force d'intensité égale, de même direction mais de sens opposé, exercée par le corps $B$.
\end{quote}

Autrement dit :

\[
\vec{F_{A/B}} = -\vec{F_{B/A}}
\]

\[
\vec{F_{A/B}} + \vec{F_{B/A}} = 0
\]

DESSIN
