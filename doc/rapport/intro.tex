\section{Introduction}

\subsection{Les moteurs physiques}

Invisibles, les phénomènes physiques qui régissent le fonctionnement
de notre univers sont pourtant omniprésents et universels. \'Etudiées
depuis des siècles, les lois les décrivant ont été à maintes reprises
redéfinies et affinées et il est de nos jours indispensable de pouvoir
les modéliser de façon fidèle ou tout du moins d'être capable de les
approximer de façon plausible.

Une simulation industrielle visant par exemple à reproduire
virtuellement les interactions entre les différentes pièces qui
composent une automobile doit être capable de reproduire de façon
réaliste la friction des pneumatiques sur le sol, l'influence de la
gravité sur le véhicule, le comportement thermique et volumique des
fluides qu'il contient ainsi que de nombreux autres aspects mécaniques
de son fonctionnement. Le système informatique simulant tous ces
facteurs est appelé moteur physique. Dans cet exemple, les enjeux de
sécurité et de qualité sont grands et des résultats d'une précision
extrême sont exigés. Les calculs à mettre en jeu pour les obtenir
peuvent donc se permettre d'être très coûteux et de se baser sur des
modélisations mathématiques complexes. Il n'est pas rare que ces
processus de simulation soient répartis sur plusieurs machines et
s'étalent sur une durée de calcul de plusieurs heures.

A contrario, les moteurs physiques d'autres types de système complexe
ne peuvent se permettre une telle latence et doivent fonctionner en
temps réel. La contrainte est encore plus forte lorsque le moteur
physique doit cohabiter avec d'autres modules gérant différents
aspects de l'application. On pense notamment aux jeux vidéo, qui
partagent le temps de calcul alloué entre le moteur graphique, le
moteur d'intelligence artificielle, la gestion du son, la gestion du
réseau, et bien sûr le moteur physique. Dans ce cadre, auquel on peut
greffer la réalité virtuelle et ses applications dérivées, la
contrainte la plus importante est le temps d'éxecution et non la
précision des résultats. On ne cherche plus à obtenir des données
exactes mais une représentation plausible du réel. Ainsi, certains
raccourcis pourront être tolérés et une part de réalisme est sacrifiée
au profit de la vitesse.

La physique est un champ très vaste dont les disciplines vont de
l'acoustique à l'électronique. Néanmoins, le spectre d'action des
moteurs physiques se limite à la mécanique classique, dite mécanique
newtonienne. Ce sous-domaine répond à des questions telles que :
Comment réagira cette balle si on la lance sur un mur ? Quelle est
l'influence d'une planète sur un objet spatial donné ? Pourquoi un
liquide visqueux dispose-t-il d'une faible vitesse d'écoulement ?
Quelles déformations engendrera un choc entre deux voitures ?

La mécanique newtonienne est elle-même une vaste discipline et la
précision d'une modélisation ne suffit pas à la différenciation entre
tous les moteurs physiques. Souvent, un moteur physique sera
spécialisé. Certains simuleront les interactions entre corps rigides
comme une boîte en plastique tombant sur le sol. Certains simuleront
le comportement d'objets déformables comme deux véhicules se
percutant. D'autres se concentreront sur les réactions entre liquides
ou entre gaz.

\subsection{Travail à accomplir}

L'objectif de ce projet est de concevoir un moteur physique de base
permettant de gérer les interactions entre des corps rigides et
convexes. On parle de corps rigides dans la mesure o\`u les objets mis
en jeu seront indéformables et incassables. Dans un tel contexte, une
tasse en porcelaine surmontée d'un bloc de granite d'une tonne ne
serait aucunement endommagée. On parle de corps convexes car se
limiter dans un premier temps à ce type de structure autorise
certaines facilités dans les calculs. Une piste pour étendre la
simulation aux solides concaves est de décomposer un corps concave en
plusieurs corps convexes.

La contrainte principale de notre application sera le temps. On veut
concevoir une simulation éxecutable en temps réel de telle façon que
si l'utilisateur modifie l'environnement de la simulation à un instant
quelconque, une réaction à cette interaction soit immédiatement
perceptible. Bien que certains raccourcis dans les calculs ainsi que
plusieurs approximations soient acceptés, le fonctionnement du moteur
se base sur des lois bien connues de la mécanique de Newton et ses
résultats ne devront pas s'éloigner de façon demesurée de ceux que
l'on retrouverait dans une situation réelle.

\subsection{\'Etude de cas}

L'activité physique d'un corps se divise en plusieurs phases. Afin de
les détailler, analysons une situation concrète. Si l'on tenait une
balle dans notre main et que nous la lâchions au dessus d'un plan,
quelles seraient les étapes que cet objet traverserait avant d'arriver
à un état de repos ?

\subsubsection{La chute}

La main s'ouvre et laisse s'échapper la balle. Notre appréhension du
monde qui nous entoure nous permet de prévoir intuitivement que
l'objet tombera et accélérera vers le bas. Ce phénomène est quantifié
par la seconde loi de Newton.
\begin{align*}
  \vec{a} = \frac{1}{m} \sum_i \vec{F}_i
\end{align*}

Cette règle décrit l'accélération $\vec{a}$ d'un corps comme étant le
produit de l'inverse de sa masse $m$ et de la somme des forces
$\vec{F}_i$ qui lui sont appliquées. Dans notre exemple, on lâche la
balle sans lui donner d'élan initial et la seule influence qu'elle
subit au cours de sa chute est celle de la gravité. La gravité
terrestre est une force de $9.81 \times m$ newtons (avec $m$ la masse de la
balle) dirigée vers le noyau de la planète mais dans la simulation, on
peut la réduire à une force dirigée vers le bas (dans la direction
négative de l'axe $y$). Afin de bénéficier d'un moteur physique
versatile, la puissance de la gravité pourra être modifiée, pour
simuler une situation lunaire par exemple, ou être totalement annulée,
pour simuler une situation d'apesanteur. En réalité, la gravité ne
sera même pas codée \og en dur \fg{} mais appartiendra à une liste de
forces environnementales qui contiendra toutes les influences que le
monde de la simulation fait subir aux corps qu'il contient. On pourra
par exemple modéliser la résistance de l'air ou la poussée irrégulière
du vent.

Cette observation du comportement de la balle nous oriente sur la
façon dont l'on pourra modéliser les déplacements d'objets soumis à
des forces extérieures. \`A chaque corps seront associées des
quantités physiques telles que la position, la vitesse et
l'accélération. Le rôle principal du moteur physique sera de faire
évoluer ces variables de façon à ce que le résultat d'une simulation
s'approche le plus possible de ce qui serait observable dans le monde
réel. Dans la première partie de ce compte-rendu, on précisera donc
les méthodes employées pour simuler la dynamique des corps.

\subsubsection{Le rebond}

Alors que la balle s'approche du plan, on s'attend naturellement à ce
qu'elle entre en contact avec ce dernier et qu'une réaction
proportionnelle à la puissance du choc soit produite. Cette réaction
dépend de nombreux facteurs, notamment des masses des objets
considérés et de leur vitesse respective.

Le moteur physique devra être capable de générer une réaction réaliste
dont la détermination passe par une formule présentée dans la seconde
partie. Néanmoins, le travail le plus complexe n'est pas de calculer
une réponse mais de détecter une collision. Plusieurs processus
géométriques devront être mis en place afin de vérifier si une
collision a lieu et si tel est le cas, afin de mesurer précisément
quels points des deux corps entrent en contact.

Une difficulté supplémentaire vient du fait que la simulation est mise
à jour de façon discrète, par pas de temps fixe. Lorsqu'une collision
sera détectée entre deux corps, il est presque impossible de se
retrouver dans une situation de contact parfait. On aura plutôt des
contacts pénétrants au sein desquels l'intégrité physique des corps
est corrompue et les objets rentrent l'un dans l'autre. Une des tâches
du moteur physique sera de pallier ce problème en recalant les corps
dans la position de contact parfait qu'ils auraient dû atteindre.

Cette phase de la vie d'un corps rigide est la plus courte,
puisqu'instantanée, mais demandera paradoxalement le plus de
travail. Les considérations géométriques qui entrent en jeu, ainsi que
les limites à contourner, imposées par l'arithmétique des nombres à
virgule flottante, seront détaillées dans la seconde section.

\subsubsection{Le repos}

Les rebonds sur le plan sont de plus en plus faibles, jusqu'à ce que
la balle n'ait plus assez d'énergie cinétique pour s'élever à
nouveau.

Son état de repos laisse penser que plus aucune force ne s'applique
sur elle. Contrairement aux apparences, dans le monde réel ce n'est
pas parce qu'un corps est fixe qu'il est libre de toute influence. La
gravité n'a pas disparu par magie et pourtant l'accélération de la
balle est nulle, puisqu'elle reste immobile. La troisième loi de
Newton \cite{newton} est là pour démystifier cette situation :

\begin{quote}
\textit{Tout corps $A$ exerçant une force sur un corps $B$ subit une force
d'intensité égale, de même direction mais de sens opposé, exercée par
le corps $B$.}
\end{quote}

Autrement dit :
\begin{align*}
  &\vec{F}_{A/B} = -\vec{F}_{B/A} \\
  &\vec{F}_{A/B} + \vec{F}_{B/A} = 0
\end{align*}

Ici, la balle subit toujours la gravité mais le plan produit une force
inverse qui permet d'annuler tout mouvement. Même si un spectateur
aura l'impression que la balle est inactive, son état collisionel
constant lui permet de rester à la surface du plan et d'équilibrer le
système. Ce phénomène sera reproductible dans le moteur physique et
sera basé sur la même méthode que les collisions classiques.

Puisque l'une des préoccupations majeures de ce projet est la vitesse
d'exécution, d'autres considérations plus techniques entreront elles
aussi en jeu. Une fois qu'un objet est immobile, par exemple, peut-on
le fixer artificiellement et ne plus faire évoluer son état pour
économiser des ressources ? Si tel est le cas, quand devra-t-on à
nouveau le rendre mobile ?
