\section{Dynamique} 

\subsection{La composante linéaire}

\subsubsection{Variables d'état}

Commençons par nous concentrer sur l'aspect cinématique d'un corps
rigide, c'est à dire son mouvement lorsqu'il n'est soumis à aucune
force extérieure. Dans un premier temps, seule la composante linéaire
du mouvement sera étudiée et les objets dont nous allons simuler le
comportement seront réduits à de simples particules. Les figures
présentées tout au long de ce rapport sont en deux dimensions pour des
raisons de clarté mais le principe reste similaire lorsqu'étendu à la
troisième dimension.

La quantité physique la plus perceptible visuellement pour un
spectateur est la position $\vec p$ d'un corps. Pour mettre en place
notre affichage final, c'est cette valeur à tout instant $t$ de la
simulation que l'on veut déterminer. Un corps possède aussi une
vitesse $\vec v$, qui correspond à la variation de sa position pour
une unité de temps. On note cette relation sous la forme dérivée :
\begin{align*}
  \vec{v} = \frac{\deriv \vec{p}}{\deriv t}
\end{align*}

Pareillement, l'accéleration $\vec a$, qui apparaissaît plus tôt dans
la seconde loi de Newton, correspond à la variation de la vitesse par
rapport à une unité de temps.
\begin{align*}
  \vec{a} = \frac{\deriv \vec{v}}{\deriv t}
\end{align*}

Par transitivité, on confirme que la position est en relation directe
avec l'accélération.
\begin{align*}
  \vec{a} = \frac{\deriv \vec{v}}{\deriv t} = \frac{\deriv^2 \vec{p}}{\deriv t}
\end{align*}

Le travail de la partie dynamique du moteur physique est de déterminer
la nouvelle position d'un objet à partir de la connaissance de ses
autres variables d'état. Grâce à l'équation précédente, on peut
calculer l'accélération d'un objet à partir de sa variation de
position. Dans le moteur physique, ce sera en fait l'opération inverse
qui devra être effectuée : on connaîtra les forces appliquées, on en
déduira par la seconde loi de Newton l'accélération induite puis on
calculera le changement de position. Il faut donc exploiter le fait
que ces trois quantités partagent aussi des relations de primitive.
\begin{align*}
  \vec{p} = \int \vec{v}\; \deriv t = \int \vec{a}\; \deriv t^2
\end{align*}

Récapitulons. Chacun des corps rigides dont l'on veut simuler
l'évolution possèdera trois quantités sous forme vectorielle :
position $\vec p$, vitesse $\vec v$ et accélération $\vec a$. L'une
des tâches de base du moteur physique sera de traduire l'application
de forces sur un corps en un changement de sa position. On sait que
les quantités énoncées entretiennent des relations de dérivation, il
faudra donc procéder par intégration pour calculer la nouvelle
position d'un objet à partir de son accélération.

\subsubsection{Intégration}

Maintenant que les trois quantités physiques entrant en jeu dans les
mouvements linéaires sont présentées, nous pouvons étudier de façon
plus concrète sur quels calculs se basera l'exercice le plus
élémentaire du moteur physique.

Les phénomènes mécaniques du monde réel évoluent de façon continue
mais notre simulation ne peut pas s'autoriser ce luxe. Le moteur
physique sera donc basé sur une simulation discrète et avancera par pas
de temps fixe $\deriv t$. \`A chaque mise à jour de la simulation, des
intégrations doivent être réalisées pour déterminer le changement
d'état d'un corps d'un instant $t_n$ à un instant $t_{n+1} = t_n +
\deriv t$. Toujours pour des raisons d'efficacité, nous ne pouvons pas
nous permettre d'allouer un temps de calcul trop important à cette
phase de la mise à jour et nous devons trouver un moyen d'approximer
ces intégrales.

Parmi les techniques classiques d'intégration approximative, on trouve
l'intégration d'Euler \cite{hecker}. Cette méthode part du principe
que l'on dispose de la valeur initiale $x_0$ de la quantité que l'on
souhaite faire évoluer ainsi que de son taux de changement $x'$ pour
une unité de temps et de la variation de temps $\deriv t$ par rapport
à l'état précédent.
\begin{align*}
  x_{n+1} = x_{n} + x' \deriv t
\end{align*}

Si l'on adapte cette méthode à notre problème, on obtient la
succession de calculs suivante :
\begin{align*}
  \vec{a}_{n+1} &= \frac{1}{m} \sum \vec{F}_i \\ \\
  \vec{v}_{n+1} &= \vec{v}_n + \vec{a}_{n+1} \deriv t \\ \\
  \vec{p}_{n+1} &= \vec{p}_n + \vec{v}_{n+1} \deriv t
\end{align*}

On calcule l'accélération à un instant $t$ puis on intègre en fonction
du temps jusqu'à obtenir la nouvelle position du corps. Ce processus
doit être repété à chaque mise à jour du système et ce, pour chaque
particule. Il est important de noter que plus $\deriv t$ sera faible
et plus les résultats seront précis. Néanmoins, le choix d'un $\deriv
t$ trop petit est susceptible de faire perdre au moteur physique son
statut d'application temps réel, puisque le programme doit être
capable de calculer $\frac{1}{\deriv t}$ itérations de la simulation
par seconde.

On aurait pu choisir une autre méthode d'intégration, telle que
l'intégration Runge-Kutta d'ordre 4 qui calcule les pentes des
subdivisions d'un pas de temps pour obtenir des résultats plus précis
\cite{fiedler}, ou bien l'intégration de Verlet qui dispose d'une
meilleure stabilité \cite{bitterli}, mais Euler reste le choix le plus
économique et fournit des résultats relativement satisfaisants.

Afin de réduire la complexité de la structure informatique qui
représentera un corps rigide et de raccourcir les calculs, nous allons
introduire la notion d'élan linéaire. Pour un corps rigide, l'élan
linéaire $\vec{L}$ est le produit de sa masse et de sa vitesse. Cette
nouvelle quantité a pour avantage majeur de posséder comme primitive
la variation de force exercée instantanément sur le corps.
\begin{align*}
  {\sum \vec{F}_i} = \frac{\deriv \vec{L}}{\deriv t} = \frac{\deriv (m\vec{v})}{\deriv t}
\end{align*}

Ce qui signifie que l'on peut réduire l'intégration de l'état d'un
corps à :
\begin{align*}
  \vec{L}_{n+1} &= \vec{L}_n + {\sum \vec{F}_i} \\ \\
  \vec{p}_{n+1} &= \vec{p}_n + \frac{1}{m}\vec{L}_{n+1} \deriv t
\end{align*}

L'élan linéaire nous débarasse de l'accélération dans la définition
d'un corps et permet de calculer sa vitesse si besoin en est.

Pour des raisons pratiques, on enregistrera dans chaque corps non pas
sa masse, mais l'inverse de celle-ci. L'avantage principal de ce choix
est de pouvoir aisément fixer des objets. Si l'on veut qu'un élément
de la simulation reste immobile, un mur fixe par exemple, on peut lui
attribuer une masse inverse nulle. De cette façon, à chaque
intégration il accumulera de l'élan mais sa position restera la
même. On élimine par la même occasion les problèmes de division par
zéro.

\subsubsection{Modélisation d'un corps}

Nous avons décrit dans la partie précédente les quantités régissant le
mouvement linéaire d'une particule ainsi que la façon dont elles
évoluent mais le moteur physique que l'on conçoit a pour visée de
simuler les comportements de corps rigides à volume convexe. Comment
peut-on étendre les principes énoncés pour des particules à ce modèle
plus complexe ?

On pourrait en premier lieu penser à représenter un tel corps par une
liste de particules, chacune placée à un sommet de l'objet. Chaque
particule évoluerait indépendement et des contraintes de cohésion
entre particules voisines seraient appliquées pour empêcher toute
déformation du corps. Cette méthode est envisageable et existe, mais
elle présente plusieurs désavantages. Premièrement, les règles de
cohésion à mettre en place nécessiteraient des traitements
supplémentaires, et donc un temps de calcul plus long. Deuxièment, un
corps devrait passer par autant d'intégrations qu'il a de particules à
chaque mise à jour. Il existe une solution plus simple et plus
élégante qui permet de réduire les mouvements d'un corps rigide à ceux
d'une unique particule judicieusement placée.

Introduisons en premier lieu la notion de repères absolu et local. Le
repère absolu est le référentiel orthonormé dont l'origine sert de
centre à l'environnement de la simulation. Un repère local est un
référentiel qui est unique à chaque corps et dont l'origine se situe à
l'intérieur même de cet objet. La position exacte de l'origine du
repère local dépend de la position des sommets qui forment l'objet
mais il ne s'agit pas d'un simple centre géométrique puisque la masse
de chaque sommet entre aussi en jeu. Cette position se nomme le centre
de masse, ou barycentre, et sera calculée dans le repère absolu par la
formule suivante, avec $M$ la masse totale des sommets du corps, $m_i$
et $\vec{p}_l$ respectivement la masse et la position du sommet $i$
dans le repère absolu :
\begin{align*}
  \vec{C} = \frac{1}{M} \sum m_i \vec{p}_i
\end{align*}

Une fois le centre de masse déterminé, la position locale $\vec{r}_i$
d'un point quelconque $i$ peut être calculée en fonction de sa
position absolue $\vec{p}_i$ par :
\begin{align*}
  \vec{r}_i = \vec{p}_i - \vec{C}
\end{align*}

La figure \ref{reperelocal} nous montre la position qu'un point $x$
d'un corps peut prendre par rapport à chaque repère : $\vec{p}_l$
correspond à sa position dans son repère local tandis que $\vec{p}_a$
est celle dans le repère absolu.

\begin{figure}
  \centering
  \begin{tikzpicture}
  % repère
\draw[->] (-0.2,0) -- (10,0);
\draw[->] (0,-0.2) -- (0,10);
\foreach \i in {0,...,9}
{
  \draw[xshift=\i cm] (0,-1pt) -- (0,1pt);
  \draw[yshift=\i cm] (-1pt,0) -- (1pt,0);
}

  % figure
  \coordinate (C) at ++(6,5);
  \draw (C) circle (2);

  % repère local  
  \begin{scope}[rotate around={-30:(C)}]
    \draw[->] (C) -- +(4,0);
    \draw[->] (C) -- +(0,4);
  \end{scope}
\end{tikzpicture}

  \caption{Les repères absolu et local d'un corps rectangulaire}
  \label{reperelocal}
\end{figure}

Le centre de masse est l'unique position que nous devons faire évoluer
par intégration, quelle que soit la complexité de la structure d'un
corps. Il reste encore néanmoins à considérer le pendant angulaire de
l'aspect dynamique d'un objet.

\subsection{La composante angulaire}

\subsubsection{Variables d'état}

Le modèle que nous avons défini est encore incomplet puisqu'il ne
prend pas en compte la composante rotationnelle des mouvements dont
l'on peut être témoin dans un environnement réel. Les particules étant
de simples points flottants dans l'espace, cela ne posait pas de
problème précédemment mais le moteur physique que l'on conçoit doit
gérer des volumes plus complexes. Imaginons une boîte cubique que l'on
lancerait devant soi, si aucune rotation n'apparaît (si la base de la
boîte reste parallèle au sol), l'imitation du réel que l'on souhaite
reproduire perd toute crédibilité.

Les quantités physiques entrant en jeu dans la décomposition d'un
déplacement angulaire sont analogues à celles présentées dans la
partie traitant de la dynamique linéaire : à la position et à l'élan
linéaire correspondent l'orientation et l'élan angulaire.

De la même façon que la position représentait visuellement l'état d'un
corps au sein de la composante linéaire, un corps doit posséder une
orientation. En deux dimensions, une valeur représentant l'angle du
corps par rapport à un axe fixe suffirait à décrire l'orientation d'un
objet mais pas dans notre environnement en trois dimensions. Le repère
local d'un corps a été introduit dans la partie précédente et se
résumait à un centre de masse faisant office d'origine, mais un repère
possède aussi des axes et ceux du repère local ne sont pas
nécessairement alignés avec ceux du repère absolu. Pour représenter la
direction des axes du repère local, et donc l'orientation du corps à
qui il appartient, on utilise une matrice $R$ de dimension 3 dans
laquelle chaque vecteur colonne correspondra à un des axes du repère
local. Pour illustrer ces propos, analysons la matrice identité
d'ordre 3, qui correspond à la matrice d'orientation d'un corps
parfaitement aligné avec les axes du repère absolu et n'ayant encore
subit aucune rotation.
\begin{align*}
  \begin{pmatrix}
    1 & 0 & 0 \\ 0 & 1 & 0 \\ 0 & 0 & 1
  \end{pmatrix}
  \rightarrow
  \begin{pmatrix}
    1 \\ 0 \\ 0
  \end{pmatrix}
  \begin{pmatrix}
    0 \\ 1 \\ 0
  \end{pmatrix}
  \begin{pmatrix}
    0 \\ 0 \\ 1
  \end{pmatrix}
\end{align*}

Si l'on isole les vecteurs colonnes de cette matrice identité, on
remarque que chacun correspond à un des axes du repère absolu. Le
premier vecteur colonne d'une matrice d'orientation correspondant à
l'axe $x$ du repère local. Il en est de même pour la seconde colonne
avec l'axe $y$ et la troisième colonne avec l'axe $z$.

La matrice d'orientation contient les directions des axes du repère
local, or on calcule la position des sommets d'un corps grâce à la
position de son centre de masse et à son orientation. Il est donc
primordial d'attacher un soin particulier à la validité de cette
matrice si l'on veut éviter toute déformation du solide ou tout
résultat incorrect. L'arithmétique des nombres à virgule flottante
impose ici son premier effet négatif. En effet, à chaque intégration
de l'état d'un corps, de légères erreurs de calcul apparaissent,
principalement sur la matrice d'orientation. L'effet est infime mais
ces erreurs s'accumulent à chaque itération, jusqu'à devenir
visuellement perceptibles. Le problème vient du fait qu'avec chaque
mise à jour de la simulation, les axes du repère local (donc les
vecteurs colonne de la matrice) perdent progressivement de leur
normalité et de leur orthogonalité. En conséquence, l'orientation
n'est pas aussi précise qu'on le souhaiterait, pire, l'objet est
gravement déformé lorsqu'on le dessine.

Afin de contrer cet effet indésirable, on ajoutera deux étapes de
correction à la fin de chaque intégration. Premièrement, il nous faut
normaliser les axes du repère local. Chaque axe correspondant à un
vecteur colonne de la matrice d'orientation, il suffit de les extraire
un par un et de recalculer leur magnitude pour s'assurer de leur
normalité. En second lieu, il sera nécessaire de réorthogonaliser le
repère local, autrement dit de s'assurer que ses axes restent
orthogonaux entre eux. Pour cela, on orthogonalisera la matrice
d'orientation par le processus de Gram-Schmidt \cite{weber}, une
méthode d'orthogonalisation fonctionnant par projection itérative des
axes les uns sur les autres.

Maintenant que l'orientation d'un corps a été définie, penchons nous
sur l'élan angulaire. On pourrait utiliser vitesse et accélération
angulaires en tant que variables d'état, mais comme pour la dynamique
linéaire on choisit de remplacer ces deux valeurs par un unique élan
angulaire. L'élan angulaire $\vec A$ possède comme primitive la
variation de force exercée sur un corps. Mais attention, sa définition
diffère de l'élan linéaire dans la mesure o\`u il n'est en relation
qu'avec la composante angulaire d'une force. En effet, il est
primordial de faire la distinction entre l'influence linéaire et
l'influence angulaire qu'une force exerce sur un corps. Quel que soit
le point d'un objet sur lequel une force est exercée, la quantité
d'élan linéaire ajoutée est la même. Par contre, la quantité d'élan
angulaire transmis par une force dépend de son point d'application;
plus précisément de son excentricité par rapport au centre de
masse. Imaginons une boîte cubique flottant en état d'apesanteur et
dont la masse est également répartie sur tous les sommets (le centre
de masse se situera donc en son centre géométrique). Si l'on exerce
une légère poussée sur le milieu d'une de ses faces alors la boîte
subira une translation. Si l'on applique maintenant une pression
toujours dans la même direction mais cette fois sur l'un des coins de
la boîte, la même translation sera accompagnée d'une rotation autour
du centre de masse. On formule la composante angulaire d'une force par
le couple $\vec{\tau}$, qui dépend de la position $\vec C$ du centre
de masse et de la position $\vec x$ du point d'application de la force
$\vec{F}$ dans le repère absolu.
\begin{align*}
  \vec{\tau} = (\vec{x} - \vec{C}) \times \vec{F}
\end{align*}

Lorsqu'une force est appliquée à un objet, elle est décomposée en sa
composante linéaire et en sa composante angulaire. Ces deux quantités
sont ensuite stockées dans des accumulateurs de force et de couple,
qui seront remis à zéro après chaque intégration. Ici, $\sum
\vec{\tau}_i$ correspond à la somme des composantes angulaires des
différentes forces $\vec{F}_i$ subies pendant une itération. Le $\sum
\vec{F}_i$ de l'intégration de la composante linéaire vue plus tôt
correspond quant à lui à la somme des composantes linéaires.
\begin{align*}
  \vec{A}_{n+1} &= \vec{A}_n + {\sum \vec{\tau}_i} \\ \\
  \vec{L}_{n+1} &= \vec{L}_n + {\sum \vec{F}_i}
\end{align*}

\subsubsection{Quantités auxiliaires}

On sait désormais qu'un corps possède une orientation et un élan
angulaire, on sait aussi comment passer de l'application de forces à
une variation d'élan angulaire. Néanmoins, l'intégration des quantités
angulaires d'un objet n'est pas aussi directe que sa version
linéaire. Nous avons encore besoin de faire appel à plusieurs
quantités auxiliaires, telles que le tenseur d'inertie local, le
tenseur d'inertie absolu et la vitesse angulaire, avant de mesurer
l'étendue de la variation d'orientation induite par des influences
extérieures.

Concrètement, un tenseur d'inertie est une matrice de dimension 3 dont
les coefficients servent de facteurs lors du calcul de la variation
d'orientation d'un corps à partir de son élan angulaire et représente
la répartition de la densité d'un corps. Il dépend directement de la
forme du corps considéré et affecte ses axes de rotation
principaux. Pour calculer un tenseur d'inertie local, on doit
effectuer des intégrations par rapport au volume du solide
considéré. Ces opérations sont un travail complexe en elles-mêmes mais
pour notre plus grande satisfaction, la plupart des manuels de
mécanique listent en annexe les tenseurs d'inertie usuels. Ci-suit, en
exemple, le tenseur d'inertie local d'une boîte, avec $d_i$ son
étendue le long de l'axe $i$ et $m$ sa masse.
\begin{align*}
  \begin{pmatrix}
    \frac{m}{12}(d_y^2 + d_z^2) & 0 & 0 \\
    0 & \frac{m}{12}(d_x^2 + d_z^2) & 0 \\
    0 & 0 & \frac{m}{12}(d_x^2 + d_y^2) \\
  \end{pmatrix}
\end{align*}

Le tenseur d'inertie local d'un corps sera attribué lors de la phase
de préparation des solides et ne changera pas au long de la
simulation, quel que soit l'état du corps.

Le tenseur d'inertie absolu est une autre quantité auxiliaire, il doit
être recalculé à chaque mise à jour puisqu'il dépend du tenseur
d'inertie local et de l'orientation actuelle du corps. La formule
suivante, avec $I_l$ le tenseur d'inertie local et $R$ la matrice
d'orientation, nous fournit le tenseur d'inertie absolu $I_{a}$.
\begin{align*}
  I_a = R I_l {}^t\!\!R
\end{align*}

On peut exprimer la vitesse angulaire $\vec{\omega}$ comme le produit
de l'inverse du tenseur d'inertie absolu et de l'élan angulaire. Cette
quantité peut être visualisée comme un vecteur dont la direction
correspond à un axe et dont la magnitude traduit le nombre de
rotations par unité de temps autour de cet axe. Il reste à lier
vitesse angulaire et matrice d'orientation. Commençons par introduire
l'opérateur $^*$ \cite{witkit}, qui transforme un vecteur en matrice,
tel que :
\begin{align*}
  \begin{pmatrix}
    x \\ y \\ z
  \end{pmatrix}
  ^*
  =
  \begin{pmatrix}
    0 & -z & y \\ z & 0 & -x \\ -y & x & 0
  \end{pmatrix}
\end{align*}

Puisque $\vec{\omega}$ correspond à une variation d'orientation le
long d'un axe, on peut écrire la variation complète d'orientation
comme :
\begin{align*}
  \frac{\deriv R}{\deriv t} =
  \begin{pmatrix}
    \vec{\omega}^* \begin{pmatrix} R_{xx} \\ R_{xy} \\ R_{xz} \end{pmatrix} & 
    \vec{\omega}^* \begin{pmatrix} R_{yx} \\ R_{yy} \\ R_{yz} \end{pmatrix} & 
    \vec{\omega}^* \begin{pmatrix} R_{zx} \\ R_{zy} \\ R_{zz} \end{pmatrix} \\
  \end{pmatrix}
\end{align*}

On peut extraire la vitesse angulaire et simplifier cette formulation
en :
\begin{align*}
  \frac{\deriv R}{\deriv t} = \vec{\omega}^*R
\end{align*}

\subsubsection{Intégration}

Toutes les quantités nécessaires à l'intégration de la composante
angulaire d'un corps sont désormais réunies.
\begin{align*}
  \vec{A}_{n+1} &= \vec{A}_n + \sum \vec{\tau}_i \\ \\
  I_a &= R_n I_l {}^t\!\!R_n \\ \\
  \vec{\omega} &= I^{-1}_a \vec{A}_{n+1} \\ \\
  R_{n+1} &= R_n + \vec{\omega}^* R_n \deriv t
\end{align*}

On part de la somme des composantes angulaires de chaque force subie
par le corps pendant une itération et on obtient le nouvel élan
angulaire. Ce dernier, utilisé conjointement avec l'inverse du tenseur
d'inertie absolu, fournit la vitesse angulaire du corps. Il reste
alors à mettre à jour la matrice d'orientation en lui ajoutant le
produit de $\vec{\omega}^*$ et de l'ancienne orientation, le tout
multiplié par le pas de temps.

Le second calcul de cette intégration nous fournit le tenseur
d'inertie absolu, mais il est encore nécessaire de l'inverser pour
calculer la vitesse angulaire. Afin de nous débarasser de cette
opération supplémentaire, un corps contiendra non pas un tenseur
d'inertie local, mais son inverse. On remplace donc ce calcul par :
\begin{align*}
  I^{-1}_a = (R_n I_l {}^t\!\!R_n)^{-1} = R_n I^{-1}_l {}^t\!\!R_n
\end{align*}
