\section{Introduction}

\subsection{Les moteurs physiques}

Invisibles, les phénomènes physiques qui régissent le fonctionnement de notre univers sont pourtant omniprésents et universels. Etudiées depuis des siècles, les lois les décrivant ont été à maintes reprises redéfinies et affinnées et il est de nos jours indispensable de pouvoir les modéliser de façon fidèle ou d'être capable de les approximer de façon plausible. 

Une simulation industrielle visant par exemple à reproduire virtuellement les interactions entre les différentes pièces qui composent une automobile doit être capable de reproduire de façon réaliste la friction des pneumatiques sur le sol, l'influence de la gravité sur le véhicule, le comportement thermique et volumique des fluides qu'il contient ainsi que de nombreux autres aspects de son fonctionnement. Le système simulant tous ces facteurs est appellé moteur physique. Dans cet exemple, les enjeux de sécurité et de qualité sont grands et des résultats d'une précision extrême sont exigés. Les calculs à mettre en jeu pour les obtenir peuvent donc se permettre d'être très coûteux et de se baser sur des modélisations mathématiques complexes. Il n'est pas rare que ces processus de simulation soient réparties sur plusieurs machines et s'étalent sur plusieurs heures.

\`A contrario, les moteurs physiques de certains systèmes complexes ne peuvent se permettre une telle latence et doivent fonctionner en temps réel. La contrainte est encore plus forte lorsque lorsque le moteur physique doit cohabiter avec d'autres modules gérant différents aspects de l'application. On pense notamment aux jeux vidéo, qui partagent le temp de calcul alloué entre le moteur graphique, le moteur d'intelligence artificielle, la gestion du son, la gestion du réseau, et bien sûr le moteur physique. Dans le cadre de ce type d'application, auquel on peut greffer la réalité virtuelle, le facteur le plus important est le temps d'éxecution et non la précision des résultats. On ne cherche plus à obtenir des données exactes mais une représentation plausible du réel. Ainsi, certains raccourcis pourront être employés et une part de réalisme est sacrifiée au profit de la vitesse.

La physique est un champ très vaste dont les disciplines vont de l'acoustique à l'électronique en passant par la physique quantique. Néanmoins, le spectre d'action des moteurs physique se limite à la mécanique classique, dite mécanique newtonienne. Ce sous-domaine répond à des questions telles que : Comment réagira cette balle si on la lance sur un mur ? Quelle est l'influence d'une planète sur un objet spatial donné ? Pourquoi un liquide visqueux dispose-t-il d'une plus faible vitesse d'écoulement ? Quelle déformation engendrera un choc entre deux voitures ?

La mécanique newtonienne est un champ large et la contrainte temporelle et la précision demandée ne sont pas des facteurs suffisants à la différenciation de tous les moteurs physiques. Souvent, un moteur physique sera spécialisé. Certains simuleront les interactions entre corps rigides, ou indéformables, comme une boîte de plastique tombant sur le sol. Certains simuleront le comportements d'objets mous, ou déformables, comme une balle de caoutchouc jeté sur un mur. D'autres se concentreront sur les réactions entre liquides ou entre gaz.

\subsection{Travail à accomplir}

L'objectif de ce projet est de concevoir un moteur physique basique permettant de gérer les interactions entre corps rigides convexes. On parle de corps rigides dans la mesure o\`u les objets mis en jeu seront indéformables et incassables. Dans une tel contexte, une tasse en porcelaine surmontée d'un bloc de granite d'une tonne ne serait aucunement endommagée. On parle de corps convexes car se limiter dans un premier temps à ce type de structure autorise certaines facilités dans les calculs. Une piste pour étendre le moteur aux solides concaves sera présentée dans la dernière partie.

La contrainte principale de notre application sera le temps. On veut concevoir une simulation éxecutable en temps réel de telle façon que si l'utilisateur modifie l'environnement de la simulation à un instant quelconque, une réaction à cette interaction soit immédiatement générée. Certains raccourcis dans les calculs ainsi que plusieurs approximations seront tolérées. Le fonctionnement du moteur se base néanmoins sur des lois biens connues de la mécanique de Newton et ses résultats ne devront s'éloigner de façon demesurée de ceux que l'on retrouverait dans la réalité.

La simulation du comportement physique d'un corps se divise en plusieurs étapes; afin de les détailler, analysons une situation concrète. Si l'on tiens une balle dans notre main et que nous la lachions au dessus d'un plan incliné, quelles seraient les phases que cet objet traversera avant d'arriver à un état de repos (id est, un état d'immobilité totale) ?

\subsubsection{La chute}

La main s'ouvre et laisse s'échapper la balle. Notre appréhension du monde qui nous entoure nous permet de prévoir sans l'ombre d'un doute que l'objet tombera et accélérera vers le bas. Ce phénomène est une conséquence direct de la Nième loi de Newton.

\cite{BLABLA je suis Newton}

Cette règle universel décrit les influences que tous les corps de l'univers provoquent entre eux. \`A une échelle terrestre, cela se traduit par le fait que les objets sont attirés par le sol, les autres influences étant négligables et minorées par cette force colossale. On peut donc simplifier cette règle par

Plusieurs des quantités fondamentales dont seront dotés nos corps comment à apparaître : accélération, vitesse, position et masse.

Le coefficient de gravitation de la Terre a pour valeur $9.81 ms^{-1}$ mais l'on souhaite qu'il puisse rester variable afin de ne pas limiter les cas d'utilisation du moteur physique. On pourrait par exemple, en changeant les variables de notre environnement, simuler des situations prenant place sur la surface lunaire.

Ces observations nous orientent sur la façon dont l'on pourra modéliser les déplacements d'objets soumis à des forces extérieures. Les quantités physiques propres à chaque corps devront évoluer constamment, de manière discrète, par pas de temps fixe. Dans la première partie de ce compte-rendu, on précisera les méthodes employés pour simuler l'aspect dynamique des objets.

\subsubsection{Le rebond}

Alors que la balle s'approche du plan incliné, on s'attend naturellement à ce qu'elle entre en contact avec ce dernier et qu'une réaction proportionnelle à la puissance du choc soit produite. Cette réaction dépend de nombreux facteurs, notamment des masses des deux objets, de leurs coefficient de restitution, de leurs coefficients de friction et de la vitesse relative de la balle par rapport au plan.

Les masses des deux corps tiennent un rôle majeur dans l'équation décrivant un rebond. Dans notre exemple le plan incliné est fixe et les masses ont une moindre influence. Par contre, si l'on imagine deux balles lancées en l'air et se rencontrant en suspension, il est naturel d'imaginer que le plus lourd des objets subira une réaction moindre.

Un coefficient de restitution $0 \leq \epsilon \leq 1$ détermine le taux de rebond d'un corps. Si $\epsilon = 1$, la balle rebondira avec autant d'énergie qu'elle est arrivée (une situation concrètement impossible à retrouver dans le monde réel). Si $\epsilon = 0$, la balle repartira avec une énergie nulle; autrement dit, elle restera collée au plan. Le coefficient de restitution dépend habituellement de la matière que l'on cherche à simuler : une balle en cahoutchouc aura un $\epsilon$ elevé tandis qu'un bloc d'argile aura un $\epsilon$ proche de zéro.

Le coefficient de friction influe sur la réponse rotationnelle qu'une collision est susceptible de provoquer. Si la balle avait été laché, à partir d'une vitesse nulle, sur un plan droit, le rebond aura été purement linéaire. Dans le cas d'un plan incliné, un frottement non perpendiculaire a lieu et la balle va entrer en rotation. Plus le coefficient de friction est élevé et plus cet effet est puissant. On peut simplifier cette notion en disant que le coefficient de friction détermine si un corps est recouvert d'une matière qui accroche ou au contraire qui glisse.

Le moteur physique devra être capable de générer des réactions réalistes en fonction de ces nombreux facteurs. La détermination de cette réponse physique passe par une formule présentée dans la seconde partie. Néanmoins, le travail le plus complexe n'est pas de calculer une réponse mais de détecter une collision. Plusieurs processus géométriques devront être mis en place afin de détecter quels points de deux corps rentrent en contact. La simulation se mettant à jour par itérations discrètes, il faudra veiller à éviter toute situation d'interpénétration et le cas échéant, à repositionner les objets.

\subsubsection{Le repos}

Les rebonds sur le plan sont de plus en plus faible, jusqu'à ce que la balle n'ait plus assez d'énergie kinétique pour s'élever à nouveau. Puisque qu'elle repose sur un plan incliné, elle roule quelques instants jusqu'à ce que la friction provoqué par le plan l'arrête totalement.

On dit alors que la balle est en état de repos. Son état immobile laisse penser que plus aucune force ne s'appliquent sur elle. Pourtant de nombreuses influences, dites de contact, permettent de conserver un état stationnaire. La gravité est toujours appliqué, et le plan produit une force inversement égal, ce qui permet à la balle de ne pas s'enfoncer dans le sol.

Dans le cadre de la simulation, il sera important de différencier les contacts rapides et les contacts de repos.

