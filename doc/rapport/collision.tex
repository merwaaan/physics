\section{Gestion des collisions}

Maintenant que la première section nous a éclairé sur la façon dont
les corps de la simulation évoluent indépendamment les uns des autres,
il est temps de les faire interagir entre eux. Cette phase se divise
en trois étapes. Premièrement, le moteur physique doit être capable de
déterminer si une collision a lieu entre deux corps. Ensuite, il doit
pouvoir corriger leur état afin de contre-balancer les décalages dûs à
l'intégration discrète de la simulation. Finalement, une force de
séparation doit être calculée et appliquée aux deux objets afin de
simuler le rebond ou le repos.

\subsection{Détection}

Une collision fait entrer en jeu deux corps rigides dont l'intégrité
physique a été corrompue; autrement dit, les corps entrent l'un dans
l'autre. Nous allons définir un traitement à appliquer à deux objets
pour savoir si tel est le cas et ce, de façon certaine. Néanmoins,
bien que l'algorithme que nous allons présenter par la suite a fait
ses preuves dans le domaine des simulations physiques, il faut garder
à l'esprit que ce test sera éxécuté à chaque intégration de la
simulation et ce, pour chaque paire de solides. Dans de telles
circonstances, l'usage d'un algorithme a priori rapide peut se révéler
désastreux pour les performances du moteur.

Nous faisons donc le choix de séparer la détection de collision en
deux étapes : une détection grossière et une détection fine. La
détection grossière fait usage d'un algorithme approximatif mais
économique qui informe de la possibilité d'une collision, si le
résultat est positif alors on exécute la détection fine pour confirmer
ou infirmer le contact de façon certaine.

\subsubsection{Détection grossière}

Le but de la phase de détection grossière (\textit{broad-phase
  collision}) est de renseigner sur la possibilité d'une collision et
ce, à moindre coût. Si une collision a réellement lieu, le résultat
sera toujours positif, néanmoins il est aussi possible que le résultat
soit positif sans qu'aucune collision ne prenne vraiment place. Dans
ce dernier cas, la détection fine invalidera la collision.

On se base sur l'utilisation de boîtes englobantes, ou \textit{AABB}
(\textit{axis-aligned bounding boxes}), pour détecter l'éventualité
d'un contact. Les boîtes englobantes sont des volumes alignés avec les
axes du repère absolu qui contiennent tous les sommets d'un
corps. Puisque tous les sommets de l'objet sont contenus dans l'AABB,
il est évident que tous les points du corps le seront aussi. Pour
savoir si deux corps rentrent possiblement en collision, on effectue
un test de collision entre leur boîte englobante.

L'algorithme \ref{algoConstructionAABB} construit la boîte englobante
d'un corps. Le principe est simple : on enregistre pour chaque axe la
position du sommet du corps le plus éloigné dans la direction négative
(\textit{min}) et la position de celui qui est le plus éloigné dans la
direction positive (\textit{max}).

\begin{algorithm}[h]
  \caption{Construction d'une AABB}
  \label{algoConstructionAABB}
  \dontprintsemicolon
  \SetKwData{axe}{axe}
  \SetKwData{sommet}{sommet}
  \Entree{Un corps C}
  \Sortie{Une boîte englobante B}
  \BlankLine
  \PourCh{\axe $A \in \{x,y,z\}$}{
    B.A.min $\leftarrow +\infty$\;
    B.A.max $\leftarrow -\infty$
  }
  \BlankLine
  \PourCh{\axe $A \in \{x,y,z\}$}{
    \PourCh{\sommet S $\in$ C}{
      \Si{S.A $<$ B.A.min}{
        B.A.min $\leftarrow$ S.A
      }
      \Si{Si S.A $>$ B.A.max}{
        B.A.max $\leftarrow$ S.A
      }
    }
  }
  \BlankLine
  \Retour{B}
\end{algorithm}

Le test vérifiant la collision entre deux boîtes englobantes est très
rapide puisque qu'il tire parti du fait que les boîtes sont alignées
avec le repère absolu. Pour vérifier que deux AABB sont en état
d'interpénétration, on utilise le théorème des axes de séparation.
Selon ce dernier, deux boîtes alignées sur les axes du repère absolu
n'entrent pas en contact si leur projection sur au moins un axe ne se
superposent pas. En deux dimensions, on peut visualiser ce théorème
par le fait que si une ligne parallèle à un des axes du repère absolu
sépare deux corps, alors ils ne peuvent pas être en état de
collision. En trois dimensions le principe est le même, à la
différence que l'on parle de plan de séparation. L'algorithme
\ref{algoDetectionGrossiere} utilise ce principe.

\begin{algorithm}[h]
  \caption{Détection grossière}
  \label{algoDetectionGrossiere}
  \dontprintsemicolon
  \SetKwData{axe}{axe}
  \SetKw{ou}{ou}
  \Entree{Deux boîtes englobantes B1 et B2}
  \Sortie{Un booléen}
  \BlankLine
  \PourCh{\axe $A \in \{x,y,z\}$}{
    \Si{B1.A.min $>$ B2.A.max \ou \\ $\;\;\;\;$ B1.A.max $<$ B2.A.min}{
      \Retour{faux}
    }
  }
  \BlankLine
  \Retour{vrai}
\end{algorithm}

La figure \ref{aabb} illustre les trois situations possibles
impliquant deux corps; un cercle et un rectangle. Sur la sous-figure
$a$, les boîtes englobantes n'entrent pas en collision car leur
projection sur l'axe des abscisses ne se superposent pas. Il est donc
impossible que les corps qu'elles contiennent soient eux-mêmes en état
de collision, il est inutile d'aller plus loin et de lancer la
détection fine.

La sous-figure $b$ montre qu'il est possible dans certaines
configurations que les boîtes englobantes entrent en collision sans
que ce soit nécessairement le cas pour les corps qu'elles
contiennent. Les boîtes sont en état d'interpénetration puisque leurs
projections ne se séparent sur aucun axe. Ici, la détection grossière
renvoie un résultat positif et c'est l'algorithme de détection fine
qui réfutera la collision entre les deux solides.

La sous-figure $c$ illustre quant à elle un troisième cas de figure
dans lequel les boîtes entrent en collision et les corps aussi. La
détection fine sera appelée et confirmera la collision.

\begin{figure}
  \centering
  \subfloat[Les boîtes ne se touchent pas, on est certain qu'aucune
    collision entre les corps n'a lieu.]{ \begin{tikzpicture}[scale=0.7, transform shape]
  % repère
\draw[->] (-0.2,0) -- (10,0);
\draw[->] (0,-0.2) -- (0,10);
\foreach \i in {0,...,9}
{
  \draw[xshift=\i cm] (0,-1pt) -- (0,1pt);
  \draw[yshift=\i cm] (-1pt,0) -- (1pt,0);
}

  \coordinate (R) at (7,3);
  \node[rectangle,rotate=-30,minimum height=3cm, minimum width=2cm,inner sep=0pt,draw=black] (rect) at (R) {};
  \coordinate (C) at (2,7);
  \node[circle,minimum size=2cm,inner sep=0pt,draw=black] (cerc) at (C) {};

  \begin{scope}[rotate around={-30:(C)}]
    \path (R) node () {}
          ++(-1,-1.5) node (a) {}
          ++(2,0) node (b) {}
          ++(0,3) node (c) {}
          ++(-2,0) node (d) {};
   \end{scope}

  % bounding boxes
  \node[fit=(a) (b) (c) (d),draw=green,dashed] {};
  \node[fit=(cerc),draw=green,dashed] {};

\end{tikzpicture}
 }
  \qquad
  \subfloat[Les boîtes se touchent et pourtant les corps ne sont pas
    en collision. La détection fine réfutera le résultat de la
    détection grossière.]{ \begin{tikzpicture}[scale=0.7, transform shape]
  % repère
\draw[->] (-0.2,0) -- (10,0);
\draw[->] (0,-0.2) -- (0,10);
\foreach \i in {0,...,9}
{
  \draw[xshift=\i cm] (0,-1pt) -- (0,1pt);
  \draw[yshift=\i cm] (-1pt,0) -- (1pt,0);
}

  \coordinate (R) at (7,3);
  \node[rectangle,rotate=-30,minimum height=3cm, minimum width=2cm,inner sep=0pt,draw=black] (rect) at (R) {};
  \coordinate (C) at (2,7);
  \node[circle,minimum size=2cm,inner sep=0pt,draw=black] (cerc) at (C) {};

  \begin{scope}[rotate around={-30:(C)}]
    \path (R) node () {}
          ++(-1,-1.5) node (a) {}
          ++(2,0) node (b) {}
          ++(0,3) node (c) {}
          ++(-2,0) node (d) {};
   \end{scope}

  % bounding boxes
  \node[fit=(a) (b) (c) (d),draw=green,dashed] {};
  \node[fit=(cerc),draw=green,dashed] {};

\end{tikzpicture}
 }
  \qquad
  \subfloat[Les boîtes se touchent et les corps qu'elles contiennent
    aussi. La détection fine validera le résultat de la détection
    grossière.]{ \begin{tikzpicture}[scale=0.7, transform shape]
  % repère
\draw[->] (-0.2,0) -- (10,0);
\draw[->] (0,-0.2) -- (0,10);
\foreach \i in {0,...,9}
{
  \draw[xshift=\i cm] (0,-1pt) -- (0,1pt);
  \draw[yshift=\i cm] (-1pt,0) -- (1pt,0);
}

  \coordinate (R) at (7,3);
  \node[rectangle,rotate=-30,minimum height=3cm, minimum width=2cm,inner sep=0pt,draw=black] (rect) at (R) {};
  \coordinate (C) at (2,7);
  \node[circle,minimum size=2cm,inner sep=0pt,draw=black] (cerc) at (C) {};

  \begin{scope}[rotate around={-30:(C)}]
    \path (R) node () {}
          ++(-1,-1.5) node (a) {}
          ++(2,0) node (b) {}
          ++(0,3) node (c) {}
          ++(-2,0) node (d) {};
   \end{scope}

  % bounding boxes
  \node[fit=(a) (b) (c) (d),draw=green,dashed] {};
  \node[fit=(cerc),draw=green,dashed] {};

\end{tikzpicture}
 }
  \caption{}
  \label{aabb}
\end{figure}

\subsubsection{Détection fine}

La phase de détection fine (\textit{narrow-phase collision}) est plus
coûteuse mais détermine de façon certaine si deux corps sont en
collision.

Introduisons en premier lieu la somme de Minkowski, une opération
mathématique notée $A \oplus B = \{a + b \mid a \in A, b \in B\}$ avec
$A$ et $B$ deux corps. On peut résumer la somme de Minkowski en un
balayage de chaque corps par l'autre. Nous utiliserons une variante de
cette opération, la différence de Minkowski, notée $A \ominus B = A
\oplus (-B)$. La propriété de cette opération qui nous intéresse le
plus est le fait que la plus petite distance entre les points qui la
forment et l'origine du repère absolu est égale à la plus petite
distance entre les corps $A$ et $B$. La figure \ref{minkowski}
illustre cette caractéristique. Nous pouvons exploiter cette
singularité pour déterminer si deux corps entrent en collision. En
effet, si la distance minimum entre les deux corps est supérieure à
zéro, alors aucune collision ne peut possiblement exister. Si au
contraire, la distance minimum est nulle, alors les deux objets sont
en état d'interpénétration.

\begin{figure}
  \centering
  \subfloat{ \begin{tikzpicture}[scale=0.5,transform shape]
  \begin{scope}[scale=0.7]
    % repère
\draw[->] (-0.2,0) -- (10,0);
\draw[->] (0,-0.2) -- (0,10);
\foreach \i in {0,...,9}
{
  \draw[xshift=\i cm] (0,-1pt) -- (0,1pt);
  \draw[yshift=\i cm] (-1pt,0) -- (1pt,0);
}
  \end{scope}

  % Rectangle
\coordinate (a1) at (4,5);
\coordinate (a2) at (4,3);
\coordinate (a3) at (6,3);
\coordinate (a4) at (6,5);

% Triangle
\coordinate (b1) at (1,2);
\coordinate (b2) at (1,1);
\coordinate (b3) at (3,1);

% Minkowski difference
\coordinate (m1) at (1,4);
\coordinate (m2) at (1,2);
\coordinate (m3) at (3,1);
\coordinate (m4) at (5,1);
\coordinate (m5) at (5,4);


  \draw[black] (a1) -- (a2) -- (a3) -- (a4) -- cycle;
  \node[black] at (5,4) (a) {$A$};

  \draw[black] (b1) -- (b2)-- (b3) -- cycle;
  \node[black] at (1.5,1.5) (b) {$B$};

  \draw[<->,bleu,thick,dashed] (a2) -- ($(b1)!(a2)!(b3)$);
\end{tikzpicture}
 }
  \subfloat{ \begin{tikzpicture}[scale=0.55,transform shape]
  \begin{scope}[scale=0.7]
    % repère
\draw[->] (-0.2,0) -- (10,0);
\draw[->] (0,-0.2) -- (0,10);
\foreach \i in {0,...,9}
{
  \draw[xshift=\i cm] (0,-1pt) -- (0,1pt);
  \draw[yshift=\i cm] (-1pt,0) -- (1pt,0);
}
  \end{scope}

  % Rectangle
\coordinate (a1) at (4,5);
\coordinate (a2) at (4,3);
\coordinate (a3) at (6,3);
\coordinate (a4) at (6,5);

% Triangle
\coordinate (b1) at (1,2);
\coordinate (b2) at (1,1);
\coordinate (b3) at (3,1);

% Minkowski difference
\coordinate (m1) at (1,4);
\coordinate (m2) at (1,2);
\coordinate (m3) at (3,1);
\coordinate (m4) at (5,1);
\coordinate (m5) at (5,4);

  
  \draw[black,thick] (m1) -- (m2) -- (m3) -- (m4) -- (m5) -- cycle;
  \node[black] at (3,3) (m) {$A \ominus B$};

  \draw[<->,vert,thick, dashed] (0,0) -- ($(m2)!(0,0)!(m3)$);
\end{tikzpicture}
 }
  \caption{Différence de Minkowski $M = A \ominus B$ entre un
    rectangle $A$ et un triangle $B$. La plus petite distance reliant
    $M$ et l'origine du répère absolu est la plus petite distance
    entre $A$ et $B$.}
  \label{minkowski}
\end{figure}

L'algorithme de détection fine consistera donc à se baser sur la
différence de Minkowski $M$ entre deux corps pour en calculer la plus
petite distance la séparant de l'origine du repère absolu. Dans cette
optique, on utilise l'algorithme \textit{GJK}, pour
\textit{Gilbert-Johnson-Keerthi algorithm}. De façon générale, cet
algorithme est capable de calculer efficacement la plus petite
distance entre la coque formée par un nuage de points et un autre
point quelconque de l'espace. Dans le cadre de notre problématique, le
nuage de points correspond aux sommets de $M$ et le point cible
correspond à l'origine du repère absolu.

GJK se base sur l'utilisation de deux éléments : le simplex et le
point de support. Un simplex d'un corps est une structure géométrique
entièrement contenue dans ce dernier et étant caractérisée par une
dimension. Un simplex de dimension 0 est un sommet, un simplex de
dimension 1 est une arête, un simplex de dimension 2 est un triangle
et un simplex de dimension 3 est un tétraèdre. Un corps de dimension
$d$ peut contenir un simplex de dimension $d$ au maximum. Au départ de
l'algorithme on part d'un simplex de base choisi alétoirement (souvent
un sommet, donc un simplex de dimension 0). Ensuite, et à chaque
étape, on augmente sa dimension pour l'agrandir jusqu'à ce que le
point de $M$ le plus proche de l'origine en fasse partie. \`A la fin
de chaque étape, on détermine le point du simplex actuellement le plus
proche de l'origine. Afin d'alléger la charge de calcul, il est aussi
nécessaire de réduire la dimension du simplex dès que des sommets ne
sont plus nécessaires à la définition du point actuellement le plus
proche. En effet, plus la dimension du simplex est faible et plus les
calculs géométriques associés à la détermination du point du simplex
actuellement le plus proche sont rapides.

Un point de support est un sommet de $M$ fourni par une fonction de
support $S(\vec{d})$, avec $\vec{d}$ la direction du point du simplex
actuellement le plus proche de l'origine à l'origine. La fonction de
support est employée dans l'algorithme principal de GJK pour obtenir
le sommet de $M$ le plus extrême dans la direction $\vec{d}$, c'est à
dire le sommet dont la projection sur $\vec{d}$ est la plus grande. Il
est à noter que grâce à la fonction de support, il n'est pas nécessaire
de calculer explicitement la différence de Minkowski \cite{ericson},
car $S_{A \ominus B}(\vec{d}) = S_A(\vec{d}) - S_B(-\vec{d})$.

La figure \ref{gjk} illustre la succession d'étapes qui amènent à la
détermination du point le plus proche de l'origine $O$

\begin{figure}[h]
  \centering
  \subfloat[]{ \begin{tikzpicture}[scale=0.7,transform shape]
	\coordinate (A) at (9,2);
\coordinate (B) at (4,1);
\coordinate (C) at (1,4);
\coordinate (D) at (2,7);
\coordinate (E) at (7,7);
\coordinate (O) at (2,2);

\node[origine] (origin) at (O) {};
\path[draw=black] (A) -- (B) -- (C) -- (D) -- (E) -- cycle;


	\node[draw=red, circle] (a) at (A) {};
  
\end{tikzpicture}
 }
  \subfloat[]{ \begin{tikzpicture}[scale=0.7,transform shape]
	\coordinate (A) at (9,2);
\coordinate (B) at (4,1);
\coordinate (C) at (1,4);
\coordinate (D) at (2,7);
\coordinate (E) at (7,7);
\coordinate (O) at (2,2);

\node[origine] (origin) at (O) {};
\path[draw=black] (A) -- (B) -- (C) -- (D) -- (E) -- cycle;


	\node[gjknode] (a) at (A) {};
	\node[gjknode] (c) at (C) {};
  
  \path[gjkpath] (a) -- (c);
\end{tikzpicture}
 }
  \qquad
  \subfloat[]{ \begin{tikzpicture}[scale=0.7,transform shape]
	\coordinate (A) at (9,2);
\coordinate (B) at (4,1);
\coordinate (C) at (1,4);
\coordinate (D) at (2,7);
\coordinate (E) at (7,7);
\coordinate (O) at (2,2);

\node[origine] (origin) at (O) {};
\path[draw=black] (A) -- (B) -- (C) -- (D) -- (E) -- cycle;


	\node[draw=red, circle] (a) at (A) {};
	\node[draw=red, circle] (c) at (C) {};
	\path[draw=green] (a) -- (c);
  
\end{tikzpicture}
 }
  \subfloat[]{ \begin{tikzpicture}[scale=0.4,transform shape]
	\coordinate (A) at (9,2);
\coordinate (B) at (4,1);
\coordinate (C) at (1,4);
\coordinate (D) at (2,7);
\coordinate (E) at (7,7);
\coordinate (O) at (2,2);

\node[origine] (origin) at (O) {};
\path[draw=black] (A) -- (B) -- (C) -- (D) -- (E) -- cycle;


	\node[gjknode] (a) at (A) {};
	\node[gjknode] (c) at (C) {};
	\node[gjknode] (b) at (B) {};

	\draw[gjkedge] (a) -- (c) -- (b) -- (a);
  \fill[rose,opacity=0.1] (A) -- (C) -- (B);

  \node[gjkclosest] (closest) at ($(C)!(O)!(B)$) {};
\end{tikzpicture}
 }
  \qquad
  \subfloat[]{ \begin{tikzpicture}[scale=0.4,transform shape]
	\coordinate (A) at (9,2);
\coordinate (B) at (4,1);
\coordinate (C) at (1,4);
\coordinate (D) at (2,7);
\coordinate (E) at (7,7);
\coordinate (O) at (2,2);

\node[origine] (origin) at (O) {};
\path[draw=black] (A) -- (B) -- (C) -- (D) -- (E) -- cycle;


	\node[gjknode] (c) at (C) {};
	\node[gjknode] (b) at (B) {};

	\path[gjkedge] (c) -- (b);
  
\end{tikzpicture}
 }
  \subfloat[]{ \begin{tikzpicture}[scale=0.4,transform shape]
	\coordinate (A) at (9,2);
\coordinate (B) at (4,1);
\coordinate (C) at (1,4);
\coordinate (D) at (2,7);
\coordinate (E) at (7,7);
\coordinate (O) at (2,2);

\node[origine] (origin) at (O) {};
\path[draw=black] (A) -- (B) -- (C) -- (D) -- (E) -- cycle;


  \node[gjkclosest] (closest) at ($(C)!(O)!(B)$) {};

  \node at ($(closest)+(0.4,0)$) (lclosest) {$S_2$};
\end{tikzpicture}
 }
  \caption{{\'E}tapes de l'algorithme GJK fournissant la plus petite
    distance entre une différence de Minkowski et le centre du repère
    absolu.}
  \label{gjk}
\end{figure}

\begin{itemize}
\item Le sommet $A$ est choisi aléatoirement pour faire office de
  simplex de base. Le point du simplex actuellement le plus proche de
  l'origine ne peut que être $A$. On cherche un point de support dans
  la direction $\vec{AO}$.
\item Le sommet le plus extrême dans la direction $\vec{AO}$ est $C$,
  on l'ajoute au simplex. Le point du simplex actuellement le plus
  proche de l'origine est $S_1$.
\item On cherche le sommet le plus extrême dans la direction
  $\vec{S_1O}$.
\item Le point de support retourné est $B$, on l'ajoute au simplex. Le
  point du simplex le plus proche de l'origine est $\vec{S_2}$.
\item Le sommet $A$ n'est plus utile à la définition de $\vec{S_2}$,
  on le retire du simplex. On cherche le sommet le plus extrême dans
  la direction $\vec{S_2O}$.
\item Aucun sommet n'est plus extrême que $\vec{S_2}$ dans la
  direction $\vec{S_2O}$, c'est donc le point de $M$ le plus proche de
  l'origine.
\end{itemize}

\subsection{Correction}

\`A chaque mise à jour de la simulation, des corps sont susceptibles
de se déplacer et donc d'entrer en contact les uns avec les autres. Le
moteur simule ces évolutions par pas de temps fixe et il est donc
improbable qu'une collision soit détectée au moment exact o\`u elle se
produit (c'est à dire à l'instant précis o\`u les deux corps sont
posés l'un contre l'autre). On est donc en permanence témoin de
situations d'interpénétration au sein desquelles deux objets rentrent
l'un dans l'autre (figure \ref{interpenetration}).  On pourrait
ignorer cette imprécision et malgré tout calculer les forces de rebond
appropriées à ce contact mais plusieurs situations critiques risquent
d'apparaître. On pense notamment au fait que la force de séparation
calculée soit sensible à cette erreur et ne déplace pas assez les
corps à l'itération suivante pour qu'ils soient séparés.

\begin{figure}
  \centering
  \begin{tikzpicture}[scale=0.7, transform shape]

\coordinate (A1) at (0,0);
\coordinate (A2) at (6.7,0);
\coordinate (B) at (8,0);

\draw node[fig,
  circle,
  minimum size=2cm,
  draw=bleu,
  dotted] (a1) at (A1) {};

\node[fig,
  circle,
  minimum size=2cm,
  draw=bleu] (a2) at (A2) {};

\node[fig,
  circle,
  minimum size=2cm,
  draw=rose] (b) at (B) {};

\draw[->,thick,vert] (A1) to node[black,auto] {$+ \deriv t$} (A2);

\end{tikzpicture}

  \caption{Situation d'interpénétration due à l'intégration discrète.}
  \label{interpenetration}
\end{figure}

Pour trouver le point de contact réel entre deux objets en collision,
il existe plusieurs approches. Les systèmes préférant favoriser le
temps d'éxécution au détriment du réalisme se contentent de
repositionner les corps à un point de contact calculé en fonction de
la profondeur de la pénétration. Mais même si la position de contact
est corrigée par cette technique, il n'en sera pas de même pour les
autres quantités physiques, notamment l'orientation, qui est plus
complexe à manipuler. Même si la vitesse d'éxecution est un facteur
important dans nos choix de conception, nous allons prendre un chemin
différent et sélectionner une méthode plus précise qui fonctionnera
par retour en arrière.

La fonction d'intégration du moteur physique prend comme seul argument
le pas de temps duquel faire avancer la simulation et pour l'instant
on utilisait un pas constant. Or, une caractéristique intéressante de
cette fonction et qu'elle peut prendre en argument un pas de temps
négatif et simuler l'évolution d'un système en sens inverse. Cette
possibilité nous permet notamment de revenir en arrière dans la
simulation dés qu'une interpénétration est détectée et ce, jusqu'à
retrouver le point de contact exact. Le terme \og exact \fg{} est à
prendre avec des pincettes puisque le moteur physique est limité par
l'arithmétique des nombres à virgule flottante et que l'on doit se
contenter d'un résultat validé par un seuil de tolérance adapté. Le
pas de temps choisi devra correspondre à une fraction du pas de temps
originel puisque le but de cette man\oe uvre est de déterminer à quel
moment du pas de temps les corps sont réellement entrés en contact.

\begin{figure}[h]
  \centering
  \begin{tikzpicture}[scale=0.7, transform shape]

\coordinate (A1) at (0,0);
\coordinate (B) at (8,0);

\coordinate (A2) at (6.7,0);
\coordinate (A2prec) at (A1);
\draw node[fig,
  circle,
  minimum size=2cm,
  draw=blue,
  dotted] (a1) at (A1) {};

\node[fig,
  circle,
  minimum size=2cm,
  draw=blue] (a2) at (A2) {};

\node[fig,
  circle,
  minimum size=2cm,
  draw=rose] (b) at (B) {};

\node[
  circle,
  dashed,
  minimum size=2.5cm,
  draw=rose] (btol) at (B) {};

\draw[->,thick,vert,font=\Large] (A2prec) to node[black,auto] {$+ \deriv t$} (A2);

\coordinate (A1) at ($(A1) + (0,-3)$);
\coordinate (A2) at ($(A2) + (0,-3)$);
\coordinate (A2prec) at ($(A2prec) + (0,-3)$);
\coordinate (B) at ($(B) + (0,-3)$);
\coordinate (A2precT) at (A2prec);
\coordinate (A2prec) at (A2);
\coordinate (A2) at ($(A2)!0.5!(A2precT)$);
\draw node[fig,
  circle,
  minimum size=2cm,
  draw=blue,
  dotted] (a1) at (A1) {};

\node[fig,
  circle,
  minimum size=2cm,
  draw=blue] (a2) at (A2) {};

\node[fig,
  circle,
  minimum size=2cm,
  draw=rose] (b) at (B) {};

\node[
  circle,
  dashed,
  minimum size=2.5cm,
  draw=rose] (btol) at (B) {};

\draw[->,thick,vert,font=\Large] (A2prec) to node[black,auto,swap] {$- \frac{\deriv t}{2}$} (A2);

\coordinate (A1) at ($(A1) + (0,-3)$);
\coordinate (A2) at ($(A2) + (0,-3)$);
\coordinate (A2prec) at ($(A2prec) + (0,-3)$);
\coordinate (B) at ($(B) + (0,-3)$);
\coordinate (A2precT) at (A2prec);
\coordinate (A2prec) at (A2);
\coordinate (A2) at ($(A2)!0.5!(A2precT)$);
\draw node[fig,
  circle,
  minimum size=2cm,
  draw=blue,
  dotted] (a1) at (A1) {};

\node[fig,
  circle,
  minimum size=2cm,
  draw=blue] (a2) at (A2) {};

\node[fig,
  circle,
  minimum size=2cm,
  draw=rose] (b) at (B) {};

\node[
  circle,
  dashed,
  minimum size=2.5cm,
  draw=rose] (btol) at (B) {};

\draw[->,thick,vert,font=\Large] (A2prec) to node[black,auto] {$+ \frac{\deriv t}{4}$} (A2);

\coordinate (A1) at ($(A1) + (0,-3)$);
\coordinate (A2) at ($(A2) + (0,-3)$);
\coordinate (A2prec) at ($(A2prec) + (0,-3)$);
\coordinate (B) at ($(B) + (0,-3)$);
\coordinate (A2precT) at (A2prec);
\coordinate (A2prec) at (A2);
\coordinate (A2) at ($(A2)!-1!(A2precT)$);
\draw node[fig,
  circle,
  minimum size=2cm,
  draw=blue,
  dotted] (a1) at (A1) {};

\node[fig,
  circle,
  minimum size=2cm,
  draw=blue] (a2) at (A2) {};

\node[fig,
  circle,
  minimum size=2cm,
  draw=rose] (b) at (B) {};

\node[
  circle,
  dashed,
  minimum size=2.5cm,
  draw=rose] (btol) at (B) {};

\draw[->,thick,vert,font=\Large] (A2prec) to node[black,auto] {$+ \frac{\deriv t}{4}$} (A2);

\coordinate (A1) at ($(A1) + (0,-3)$);
\coordinate (A2) at ($(A2) + (0,-3)$);
\coordinate (A2prec) at ($(A2prec) + (0,-3)$);
\coordinate (B) at ($(B) + (0,-3)$);
\coordinate (A2precT) at (A2prec);
\coordinate (A2prec) at (A2);
\coordinate (A2) at ($(A2)!0.5!(A2precT)$);
\draw node[fig,
  circle,
  minimum size=2cm,
  draw=blue,
  dotted] (a1) at (A1) {};

\node[fig,
  circle,
  minimum size=2cm,
  draw=blue] (a2) at (A2) {};

\node[fig,
  circle,
  minimum size=2cm,
  draw=rose] (b) at (B) {};

\node[
  circle,
  dashed,
  minimum size=2.5cm,
  draw=rose] (btol) at (B) {};

\draw[->,thick,vert,font=\Large] (A2prec) to node[black,auto,swap] {$- \frac{\deriv t}{8}$} (A2);

\end{tikzpicture}

  \caption{Correction dichotomique de la configuration de contact. Le cercle en tirets correspond au seuil de tolérance.}
  \label{dichotomie}
\end{figure}

Usuellement, ce retour en arrière s'effectue en intégrant plusieurs
fois la simulation par un pas de temps négatif et fixe, par exemple
$-\frac{\deriv t}{10}$. Néanmoins cette méthode s'adapte mal aux
simulations contenant des objets évoluant à haute vitesse puisqu'en
utilisant cette même fraction du pas de temps, un corps s'étant
déplacé de dix mètres pendant l'intégration sera au moins recalé un
mètre avant tout contact réel, même si la profondeur de pénétration
n'était que de quelques centimètres.

Afin d'accélérer cette recherche et d'obtenir des résultats précis, on
fait le choix de procéder par dichotomie. La recherche se compose de
deux phases qui s'alternent jusqu'à la découverte d'une solution : la
phase de recul consiste en un retour en arrière dans le temps tandis
que la phase d'approche est un avancement. On commence la recherche
par la phase de recul. On passe de recul à approche lorsque les corps
n'entrent plus en collision. On passe d'approche à recul lorsque les
corps entrent à nouveau en collision. Le sous-pas de temps employé est
divisé par deux à chaque changement de phase. On considère avoir
trouvé une configuration satisfaisante lorsqu'il n'y a plus
d'interpénétration et que la distance entre les objets est inférieure
à un seuil de tolérance prédéfini. Ce processus est visible sur la
figure \ref{dichotomie} et est détaillé dans l'algorithme
\ref{algoDichotomie}.

\begin{algorithm}
  \caption{Correction d'une collision}
  \label{algoDichotomie}
  \dontprintsemicolon
  \SetKw{et}{et}
  \SetKwData{RECUL}{RECUL}
  \SetKwData{APPR}{APPR}
  \SetKwFunction{distance}{distanceGJK}
  \SetKwFunction{integrer}{integrer}
  \SetKwFunction{correction}{correction}
  \Entree{
    Deux corps A et B,\\
    $\;\;\;\;\;\;\;\;\;\;\;\;\;\;\;\;\;\;\;\;\;\;\;$ une phase $P \in \{ \RECUL, \APPR \}$\\
    $\;\;\;\;\;\;\;\;\;\;\;\;\;\;\;\;\;\;\;\;\;\;\;$ un pas de temps $\deriv t$,\\
    $\;\;\;\;\;\;\;\;\;\;\;\;\;\;\;\;\;\;\;\;\;\;\;$ un seuil de tolérance $\theta$}
  \Sortie{Deux corps A et B corrigés}
  \BlankLine
  $d =$ \distance{A,B}\;
  \BlankLine
  \Si{$0 < d \leq \theta$}{
    \Retour{A \et B}
  }
  \BlankLine
  \SinonSi{$d = 0$}{
    \lSi{$P \ne \RECUL$}{
      $\deriv t \leftarrow \frac{\deriv t}{2}$
    }
    \BlankLine
    \integrer{A,$- \deriv t$}\;
    \integrer{B,$- \deriv t$}\;
    \correction{A, B, \RECUL, $\deriv t$, $\theta$}
  }
  \BlankLine
  \Sinon{
    \lSi{$P \ne \APPR$}{
      $\deriv t \leftarrow \frac{\deriv t}{2}$
    }
    \BlankLine
    \integrer{A,$+ \deriv t$}\;
    \integrer{B,$+ \deriv t$}\;
    \correction{A, B, \APPR, $\deriv t$, $\theta$}
  }
\end{algorithm}

Une fois ce processus achevé, la configuration de contact est
retrouvée et les deux corps ne sont plus en situation de pénétration
mutuelle. On remarque que grâce à cette technique, la position n'est
pas la seule valeur à avoir été recalée : orientation, élans angulaire
et linéaires sont eux aussi revenus aux valeurs qu'ils auraient dû
atteindre à l'instant du contact.

\subsection{Réponse}

Une fois que la collision entre deux solides est confirmée et que leur
état est corrigé, il reste à rassembler toutes les informations
nécessaires au calcul d'une réponse. Pour cela, on aura besoin de
déterminer quels points précis des corps entrent en contact.  Parfois,
un seul point de contact peut exister, comme lors d'une collision
entre deux sphères. Souvent, il en existera plusieurs. Un corps rigide
est structuré par trois types d'élément : les sommets, les arêtes et
les faces et il est usuellement admis que seuls les contacts
sommet-face et arête-arête sont nécessaires à nos besoins. Les autres
associations d'éléments peuvent être ramenées à des configurations
dégénérées de ces deux cas. Par exemple, un contact entre une arête et
une face peut être assimilée à deux contacts sommet-face, chaque
sommet étant une extrémité de l'arête. La figure \ref{contacts}
détaille les points de contact intervenant dans une collision entre
deux cubes.

\begin{figure}
  \centering
  \subfloat{
    \begin{tikzpicture}[scale=0.4]
  % Avant.
  \begin{scope}[canvas is xy plane at z=4]
    \path[fill=gris!20] (0,0) rectangle (4,4);
  \end{scope}
  % Droite.
  \begin{scope}[canvas is zy plane at x=4]
    \path[fill=gris!50] (0,0) rectangle (4,4);
  \end{scope}
  % Haut.
  \begin{scope}[canvas is zx plane at y=4]
    \path[fill=gris!70] (0,0) rectangle (4,4);
  \end{scope}

  % Avant.
  \begin{scope}[canvas is xy plane at z=2]
    \path[fill=gris!20] (2,4) rectangle (6,8);
  \end{scope}
  % Droite.
  \begin{scope}[canvas is yz plane at x=6]
    \path[fill=gris!50] (4,2) rectangle (8,-2);
  \end{scope}
  % Haut.
  \begin{scope}[canvas is xz plane at y=8]
    \path[fill=gris!70] (2,2) rectangle (6,-2);
  \end{scope}
\end{tikzpicture}

    w\begin{tikzpicture}[scale=0.4]
  % Avant.
  \begin{scope}[canvas is xy plane at z=4]
    \path[draw=black, thick] (0,0) rectangle (4,4);
  \end{scope}
  % Droite.
  \begin{scope}[canvas is zy plane at x=4]
    \path[draw=black, thick] (0,0) rectangle (4,4);
  \end{scope}
  % Arrière.
  \begin{scope}[canvas is xy plane at z=0]
    \path[draw=black,dashed,thin] (0,0) rectangle (4,4);
  \end{scope}
  % Gauche.
  \begin{scope}[canvas is yz plane at x=0]
    \path[draw=black,dashed,thin] (0,0) rectangle (4,4);
  \end{scope}
  % Haut.
  \begin{scope}[canvas is zx plane at y=4]
    \path[draw=black, thick] (0,0) rectangle (4,4);
  \end{scope}

  % Avant.
  \begin{scope}[canvas is xy plane at z=2]
    \path[draw=black, thick] (2,4) rectangle (6,8);
  \end{scope}
  % Droite.
  \begin{scope}[canvas is yz plane at x=6]
    \path[draw=black, thick] (4,2) rectangle (8,-2);
  \end{scope}
  % Haut.
  \begin{scope}[canvas is xz plane at y=8]
    \path[draw=black, thick] (2,2) rectangle (6,-2);
  \end{scope}
  % Arrière.
  \begin{scope}[canvas is xy plane at z=-2]
    \path[draw=black,dashed,thin] (2,4) rectangle (6,8);
  \end{scope}
  % Gauche.
  \begin{scope}[canvas is yz plane at x=2]
    \path[draw=black,dashed,thin] (4,2) rectangle (8,-2);
  \end{scope}

  \begin{scope}[canvas is xz plane at y=4]
    \node[fill=bleu,circle] at (2,0) (c1) {};
    \node[fill=rose,circle] at (2,2) (c2) {};
    \node[fill=bleu,circle] at (4,2) (c3) {};    
    \node[fill=rose,circle] at (4,0) (c4) {};
  \end{scope}
\end{tikzpicture}

  }
  \caption{Quatre points de contact entre deux cubes : deux
    sommet-face (en rose) et deux arête-arête (en bleu).}
  \label{contacts}
\end{figure}

Pour énumérer les points de contact entre deux corps, on effectue des
tests de proximité entre chaque paire sommet-face et
arête-arête. Cette phase repose sur plusieurs procédures géométriques
utilitaires qui déterminent la plus faible distance entre deux
éléments par projection. On considère que deux éléments sont en
contact si leur distance est inférieure à un seuil fixé qui doit être
en accord avec celui utilisé lors de la correction de l'état
collisionnel.

Nous référerons désormais aux informations concernant un point de
contact entre deux corps en tant que \textit{contact}. Un contact ne
s'agit pas uniquement d'une unique position à laquelle deux corps se
touchent, il doit aussi renseigner sur la direction de la collision et
contiendra donc un vecteur normal $\vec{n}$ qui variera selon le type
de contact. Pour un contact sommet-face, il s'agira d'un vecteur
normal à la face et dirigée vers le sommet. Pour un contact
arête-arête, il s'agira d'un vecteur unité orthogonal aux deux arêtes,
et donc de direction $\vec{a}_1 \times \vec{a}_2$, avec $\vec{a}_1$
(respectivement $\vec{a}_2$) le vecteur reliant les deux extrémités de
la première (respectivement seconde) arête. Pour des raisons
pratiques, on y adjoindra aussi une référence vers chacun des corps
entrant en jeu dans la collision. Il reste une information
supplémentaire à enregistrer : le temps de contact. Par temps de
contact, on ne pense pas à la durée d'un contact, puisqu'ils sont
instantanés dans cette simulation, mais au moment précis du pas de
temps auquel le contact est apparu. On pourrait penser que, comme le
pas de temps de la simulation est fixe, tous les contacts inexistants
au temps $t$ et apparus au temps $t + \deriv t$ auront un temps
d'impact $\deriv t$ mais la routine de correction de l'état
collisionnel doit être capable de cibler précisément le moment du
contact. Cette information se révélera utile dans la dernière partie
de ce rapport, lorsque l'on souhaitera assurer le bon ordonnancement
des collisions.

Après chaque intégration et si deux corps entrent en collision, des
forces de séparation devront être générées pour annuler cette
collision, on parle d'\textit{impulsions} \cite{mirtich}. Une
collision provoquera autant d'impulsions qu'il existe de contacts
entre les deux objets considérés. Dans le cas d'une boîte cubique
tombant à plat sur le sol, quatre points de contact seront détectés :
les quatres sommets de la face basse du cube. \`A chacune de ces
positions, une impulsion identique sera appliquée. Dans ce cas précis,
on imagine que la masse de la boîte est également répartie, les
quatre impulsions devront donc être égales et dirigées vers le haut
afin de ne pas déséquilibrer l'objet et de produire un rebond droit.

Pour calculer une impulsion $\vec{J}$ séparant deux corps $A$ et $B$, on
utilise cette intimidante formule d'Isaac Newton, tirée de sa loi de
la restitution des collisions instantanées sans friction :
\begin{align*}
\vec{J} = \vec{n} 
  \frac{-(1 + \varepsilon) v_r}{
  \frac{1}{m_A} +
  \frac{1}{m_B} +
  \vec{n}
  (I_A^{-1} (\vec{r}_A \times \vec{n})) \times \vec{r}_A +
  (I_B^{-1} (\vec{r}_b \times \vec{n})) \times \vec{r}_B
}
\end{align*}

On y voit apparaître plusieurs quantités présentées plus tôt,
notamment les tenseurs d'inertie absolus de chaque corps impliqué dans
la collision ainsi que leur masse. On observe que le vecteur normal
$\vec{n}$ enregistré lors de la recherche des points de contact y joue
un rôle important puisqu'au final, l'impulsion correspondra au produit
de $\vec{n}$ et d'un scalaire. Plusieurs autres valeurs sont néanmoins
encore inconnues et doivent être calculées à partir des informations
de contact dont l'on dispose.

Les vecteurs $\vec{r}_A$ et $\vec{r}_B$ correspondent aux positions du
point de contact dans les référentiels respectifs des corps $A$ et $B$.

La vitesse relative normale $v_r$ correspond à la composante de la
vitesse relative des deux corps le long de la normale du contact. On
peut calculer la vitesse $\vec{v}_l$ d'un point quelconque du repère
absolu dans le repère local d'un corps avec la vitesse linéaire
$\vec{v}$ du corps, sa vitesse angulaire $\vec{\omega}$, la position
$\vec{C}$ de son centre de masse et la position $\vec{p}_a$ du point
dans le repère absolu.
\begin{align*}
  \vec{v}_l &= \vec{v} + \vec{\omega} \times (\vec{p}_a - \vec{C}) \\
            &= \frac{1}{m} \vec{L} + (I^{-1}_a \vec{A}) \times \vec{r}
\end{align*}

Pour obtenir, la vitesse relative normale, il reste à projeter sur la
normale du contact considéré la vitesse relative du point de contact
par rapport aux référentiels des deux corps. Ici, $\vec{v}_a$ et
$\vec{v}_b$ correspondent aux vitesses du point de contact par rapport
aux repères locaux de $A$ et de $B$.
\begin{align*}
  v_r &= \vec{n} (\vec{v}_{A} - \vec{v}_{B})
\end{align*}

Le coefficient de restitution $\varepsilon$, avec $0 \leq \varepsilon
\leq 1$, détermine le taux de rebond d'un corps. Reprenons l'exemple
de la balle de caoutchouc utilisé en guise d'étude de cas. Si
$\varepsilon = 1$, la balle rebondira avec autant d'énergie qu'elle
est arrivée; une situation concrètement impossible à retrouver dans le
monde réel. Si $\varepsilon = 0$, la balle repartira avec une énergie
nulle; autrement dit, elle restera collée au plan. Le coefficient de
restitution dépend habituellement de la matière que l'on cherche à
simuler : une balle en caoutchouc aura un $\varepsilon$ elevé tandis
qu'un bloc d'argile aura un $\varepsilon$ proche de zéro. Dans le
moteur physique, chaque objet possède son propre coefficient de
restitution et on choisit d'utiliser le minimum des deux dans la
formule de Newton.

Une fois l'impulsion calculée, il reste à l'appliquer à un des deux
corps impliqué dans la collision et à appliquer son opposée à l'autre
corps pour les séparer.

Il est encore nécessaire de traîter le cas des contacts de repos,
c'est à dire les collisions entre solides posés l'un contre l'autre,
comme par exemple une boîte placée sur la surface d'une table. Nous
avons évoqué plus tôt le fait que les contacts de repos sont analogues
aux contacts classiques dans la mesure o\`u la séparation des deux
corps sera exécutée par des impulsions. La seule différence vient du
fait que dans ce cas, on forcera l'utilisation d'un coefficient de
restitution nul, afin de produire des impulsions
non-élastiques. Autrement dit, les objets seront séparés afin de
d'éliminer toute interpénétration mais aucun rebond supplémentaire ne
sera injecté dans la formule. Pour détecter s'il est nécessaire
d'appliquer un tel traitement, on vérifie si la vitesse relative
normale des objets considérés est inférieure à un seuil fixé.

Bien que ce choix permette de simuler le repos d'un corps, il souffre
de défauts majeurs, notamment le fait qu'à cause de l'utilisation
d'impulsions, si faibles soit-elles, on remarque un léger effet de
vibration provoqué par les forces qu'un corps subit continuellement
pour rester à la surface d'une autre. Parfois, un corps au repos
accumulera assez d'énergie pour qu'un faible rebond soit
perceptible. Ce type d'occurence appauvrit grandement la stabilité du
système. Pour les contrer, on désactive temporairement les objets en
état de repos; on parle de mise en sommeil. L'avantage est double
puisqu'en plus d'améliorer la stabilité générale, on économise les
ressources car les corps endormis ne sont plus intégrés. Pour
déterminer si un objet doit être mis en sommeil, on observe sa
quantité d'énergie cinétique. L'énergie $E_i$ d'une particule $i$ est
quantifiable, avec $m$ sa masse et $\vec{v}_l$ sa vitesse relative,
par :
\begin{align*}
  E_i = \frac{1}{2} m \vec{v}^{\;2}_l
\end{align*}

On choisit d'examiner cette quantité car la vitesse relative dont elle
dépend comprend les composantes linéaire et angulaire de sa
vitesse. Pour connaître l'énergie cinétique d'un corps, on calcule la
somme des énergies cinétiques de ses sommets.
\begin{align*}
  E = \sum_i E_i
\end{align*}

On ne peut pas se baser sur ce facteur de façon instantanée car un
solide projeté en hauteur possédera d'une énergie cinétique nulle au
sommet de sa trajectoire et on ne voudrait pas l'endormir à ce
moment. On surveille plutôt son énergie pendant un nombre fixé de
mises à jour et on l'endort si elle reste faible assez longtemps. On
réveille un corps quand un objet actif entre en collision avec ce
dernier.
