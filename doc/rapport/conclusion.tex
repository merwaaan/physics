\section{Conclusion}

Le but de ce projet était d'étudier et d'implémenter un moteur
physique de base simulant le comportement mécanique de corps
rigides. Pour ce faire, nous nous sommes basés sur une modélisation
des lois de Newton utilisant forces et impulsions. De nombreux choix
concernant les solutions utilisées sont justifiés par la contrainte du
temps réel. Désirant néanmoins une modélisation aussi précise que
possible, il a été nécessaire d'implémenter des méthodes de correction
permettant de conserver à la fois cohérence spatiale et cohérence
temporelle.

Le problème récurrent de ce type d'application est la stabilité, non
pas au sens de la stabilité logicielle, mais plutôt en celui de la
stabilité des configurations atteintes. Afin d'obtenir des simulations
équilibrées, des seuils de tolérance calibrés ont dû être employés à
différentes étapes du travail du moteur physique, comme lors de la
phase de détection de collision et de correction. \`A l'heure
actuelle, il est tout de même difficile d'obtenir certaines
configurations, notamment des empilement de solides, cas dans lequel
les collisions de tous les objets au repos apparaissent au même
instant. Une solution serait de mettre en place une technique de
correction indépendante du temps. Elle déterminerait la profondeur de
pénétration associée à une collision et recalerait les objets en
modifiant uniquement leur position. On perdrait la cohérence
temporelle et on ignorerait les variations d'orientation entre l'état
au moment de la détection et l'état après correction, mais cette
approximation augmenterait les capacités du moteur physique tout en
conservant l'atout du temps réel.

D'autres modélisations existent et sont fondées sur la résolution de
systèmes linéaires pour gérer les contacts. \`A chaque mise à jour, on
converge progressivement vers une configuration satisfaisante. Il
n'est dans ce cas pas gênant que des interpénétrations existent tant
qu'elles restent imperceptibles. La stabilité fourni par ce genre de
méthode justifie à elle seule l'emploi de telles techniques. Il serait
intéressant de les étudier à l'avenir.
