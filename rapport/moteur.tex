\section{Le moteur physique}

\subsection{Ordre des tâches}

\subsection{Démonstrations}

\subsection{Limitations et perspectives d'évolution}

\subsubsection{Tunneling}

Le moteur physique que nous concevons fonctionne par intégration
discrète et si aucun contrôle n'est effectué pour contrecarrer les
effets négatifs de cette caractéristique, on risque d'obtenir des
résultats imprécis ou pire, complètement éloignés de ce que l'on
retrouverait dans la réalité. On a par exemple vu plus tôt qu'une
phase de recherche dichotomique est nécessaire pour recaler les corps
à leur position réelle de contact lorsqu'une collision est détectée.

Néanmoins, une éventualité n'a pas été envisagée : que se passerait-il
si un objet en traversait entièrement un autre pendant un pas de temps
? On peut imaginer un objet $A$ lancé vers un autre objet $B$
fixe. \`A l'instant $t$, $A$ n'entre pas en collision avec $B$ mais se
dirige dans sa direction. \`A l'instant $t+1$, $A$ a entièrement
traversé $B$. \`A aucun de ces deux instants une collision n'a été
détectée et pourtant $A$ est impunément passé à travers $B$. Le pas de
temps utilisé pour réguler l'intégration du système était donc trop
faible pour s'assurer des collisions entre corps évoluant à haute
vitesse. \`A l'heure de l'écriture de ce rapport, le moteur physique
est sensible à ce genre de situation. Réellement, ce cas ne se produit
pas, puisque les simulations développées représentent des situations
terrestres qui font participer des forces d'opposition telles que le
frottement de l'air et donc les corps n'atteignent jamais de vitesses
démesurées. On souhaite tout de même disposer d'un moteur physique
solide et versatile, évaluons les possibilités qui s'offrent à nous
pour pallier ce problème.

\begin{figure}
  \centering
  \begin{tikzpicture}[scale=0.7, transform shape]
  \coordinate (AT1) at (0,0);
\coordinate (AT2) at (8,0);
\coordinate (B) at (6,0);

\draw node[fig,
  circle,
  minimum size=2cm,
  draw=bleu,
  dotted] (a1) at (AT1) {};

\node[fig,
  circle,
  minimum size=2cm,
  draw=bleu] (a2) at (AT2) {};
\node[fig,
  circle,
  minimum size=1cm,
  draw=rose] (b) at (B) {};



  \draw[->,thick,vert] (AT1) to node[black,auto] {$\deriv t$} (AT2);
\end{tikzpicture}

  \caption{}
  \label{tunneling1}
\end{figure}

On pourrait envisager diminuer la durée du pas de temps d'intégration
afin de diminuer le risque de tunneling, mais même si un pas de temps
faible améliore la qualité des résultats, il ne pourra pas être
indéfiniment réduit. Il est principalement limité par le temps de
calcul d'une mise à jour du système, puisqu'à partir du moment o\`u
$\deriv t$ est inférieur au temps moyen nécessaire à une machine pour
calculer une itération de la simulation, le programme perdra son
statut d'application temps réel.

Une technique usuellement admise est le lancer de rayons
(\textit{raycasting}) entre les points de la position de départ du
solide et les points de sa position d'arrivée. Si l'un des rayons
touche un autre corps, on sait qu'une collision aurait dûe être
détectée et on peut revenir en arrière. Cette méthode présente
pourtant une faiblesse de taille, car le résultat de cette recherche
dépend directement de la concentration de rayons. Si on lance peu de
rayons, on risque de ne pas détecter les corps de petite taille et
donc de les traverser tandis qu'une concentration élevée de rayons
pourrait se réveler regrettable en terme de coûts de calcul.

\begin{figure}
  \centering
  \begin{tikzpicture}[scale=0.7, transform shape]
  \coordinate (AT1) at (0,0);
\coordinate (AT2) at (8,0);
\coordinate (B) at (6,0);

\draw node[fig,
  circle,
  minimum size=2cm,
  draw=bleu,
  dotted] (a1) at (AT1) {};

\node[fig,
  circle,
  minimum size=2cm,
  draw=bleu] (a2) at (AT2) {};
\node[fig,
  circle,
  minimum size=1cm,
  draw=rose] (b) at (B) {};



  \path[rayon] (a1.north) -- (a2.north);
  \path[rayon] (a1.west) -- (a2.west);
  \path[rayon] (a1.north west) -- (a2.north west);
  \path[rayon] (a1.south) -- (a2.south);
  \path[rayon] (a1.south west) -- (a2.south west);
\end{tikzpicture}

  \caption{}
  \label{tunneling2}
\end{figure}

La technique du ??? permet de contrôler de façon certain tout
phénomène de tunneling. L'idée est la suivante : soit deux corps $A$
et $B$ dont l'on veut détecter l'éventuelle collision. On construit
deux volumes fantômes $F_A$ et $F_B$ englobant tous les points par
lesquels passent respectivement $A$ et $B$ entre deux
intégrations. Ces volumes n'interviendront pas dans les collisions des
autres objets de la simulation, et on vérifie simplement s'ils
rentrent tous deux en collision. Si tel est le cas, alors on sait
qu'il est possible qu'un tunneling se soit produit. L'un des avantages
majeurs de cette méthode, en plus du fait qu'elle ne rate aucun
tunneling, est sa simplicité de mise en place puisque toutes les
routines utilisées (détection de collision entre volumes, retour en
arrière) ont été écrites et entrent déjà en jeu dans le fonctionnement
de base du moteur physique.

\begin{figure}
  \centering
  \begin{tikzpicture}[scale=0.7, transform shape]
  \coordinate (AT1) at (0,0);
\coordinate (AT2) at (8,0);
\coordinate (B) at (6,0);

\draw node[fig,
  circle,
  minimum size=2cm,
  draw=bleu,
  dotted] (a1) at (AT1) {};

\node[fig,
  circle,
  minimum size=2cm,
  draw=bleu] (a2) at (AT2) {};
\node[fig,
  circle,
  minimum size=1cm,
  draw=rose] (b) at (B) {};



  \coordinate (a1n) at ($(a1.north)+(0,0.2)$);
  \coordinate (a2s) at ($(a2.south)+(0,-0.2)$);

  \draw[gris] (a1n) arc (90:270:1.2cm) --
  (a2s) arc (-90:90:1.2cm) --
  (a1n);
\end{tikzpicture}

  \caption{}
  \label{tunneling3}
\end{figure}

\subsubsection{Partitionnement de l'espace}

Comme mis en avant tout au long de ce rapport, on cherche à concevoir
un moteur physique dont la rapidité d'éxécution est un facteur
déterminant. Dans cette optique, il est primordial d'optimiser les
routines géométriques entrant en jeu mais aussi d'éviter leur
éxécution inutile dés que possible. La détection des collisions entre
les corps qui peuplent notre système se déroule en deux étapes : une
détection grossière visant à vérifier de façon économique si deux
objets se touchent et une détection fine servant à confirmer ou à
infirmer de façon sûre ce résultat. Bien que la détection grossière
soit peu coûteuse en termes de calcul, elle est éxécutée $???$ à
chaque itération, o\`u $n$ est le nombre de corps actifs dans la
simulation. Pourtant, il est très souvent inutile de détecter si deux
objets sont en situation d'interpénétration, par exemple lorsqu'ils
sont très éloignés l'un de l'autre. Il serait tout à notre avantage
d'ajouter au moteur une couche supplémentaire tenant compte de ces
disparités spatiales pour accélérer la détection de collisions.

Le principe général du partitionnement de l'espace (\textit{Spatial
  partitioning}) est de subdiviser l'environnement de la simulation en
sous-espaces. La répartition des corps dans ces sous-espaces dépend
bien évidemment de leur position dans le repère global mais aussi du
volume qu'ils occupent. On est certain que seules les entités
appartenant à un même sous-espace peuvent interagir entres elles, on
dispose donc d'une aide quant au choix des objets entre lesquels
vérifier si collision il y a.

Sous sa forme la plus simple, le partitionnement de l'espace prend la
forme d'une grille de taille fixe dans les cases de laquelle les corps
de la simulation sont répartis. L'aspect statique d'une telle
structure présente plusieurs inconvénients majeurs, notamment le fait
que le gain de performance variera fortement selon le nombre d'objets
considérés, leur taille et la finesse de la grille. Par exemple, si
une grille a des cases trop grandes, chacune contiendra de nombreux
objets, et le gain de performance sera peu flagrant. Si au contraire
les cases de la grille sont trop petites, les plus grands objets ne
pourront pas rentrer dans leur intégralité à l'intérieur d'une unique
case. Une solution à ce problème est de ranger un même objet dans
plusieurs cases différentes de la grille, mais une fois encore, si des
corps sont démultipliés de la sorte, le temps de maintenance d'une
telle structure sera augmentée.

Une variante plus subtile de partitionnement de l'espace est le
partionnement binaire blabla.

Quatree, octree, blabla

\subsubsection{Extension aux corps concaves}


