\section{Gestion des collisions}

Maintenant que la première section nous a éclairé sur la façon dont les corps de la simulation évoluent indépendament les uns des autres, il est temps de les faire interagir entre eux. Cette phase se divise en trois étapes. Premièrement, le moteur physique doit être capable de déterminer si une collision a lieu entre deux corps. Ensuite, il doit pouvoir corriger leurs états afin de contre-balancer les décalages dû aux erreurs numériques et les limitations imposées par l'intégration discrète de l'environnement. Finalement, une force de séparation doit être calculée et appliquée aux deux objets.

\subsection{Détection d'une collision}

\subsubsection{Détection fine}



\subsubsection{Détection grossière}


\subsection{Correction d'une collision}

\`A chaque itération de la simulation, les corps présents dans l'environnement du moteur physique sont susceptibles de se déplacer et donc d'entrer en contact. Le moteur simule ces évolutions de manière non continue et il est très improbable qu'une collision soit détectée au moment précis o\`u elle se produit. On est donc régulièrement témoin de situations d'inter-pénétration au sein desquelles des corps rentrent l'un dans l'autre.

On pourrait se satisfaire de cette imprécision et calculer les forces de rebond appropriées mais plusieurs situations critiques risquent d'apparaître. La force de séparation génération ne pourrait pas être suffisante pour séparer entièrement les deux sphères, ce qui les bloqueraient balbla. Pour trouver le point de contact réel entre deux sphères, il existe plusieurs approches. Les systèmes qui préfèrent favoriser le temps d'éxécution au détriment du réalisme se contente de repositionner les corps à un point de contact calculé par interpolation en fonction du temps de la dernière ntégration. Il est évident que des objets sujets à des déplacements non linéaires (variation de la vitesse pendant un pas de temps, trajectoire elliptique) seront mal replacés.

Même si la vitesse d'éxecution est un facteur important dans nos choix de conception, nous allons prendre un chemin différent et sélectionner une méthode plus précise qui fonctionnera par retour en arrière. La fonction d'intégration du moteur prend comme seul argument le pas de temps duquel faire avancer la simulation et pour l'instant on utilisait un pas constant. Or, une caractéristique intéressante de cette fonction et qu'elle peut prendre en argument un pas de temps négatif et simuler en sens inverse ! Grâce à cette caractéristique, dés qu'une inter-pénétration est détectée, on peut revenir en arrière jusqu'à retrouver le temps de contact exact. Le terme "exact" est à prendre avec des pincettes puisque notre simulation est limité par l'arithmétique des nombres à virgules flottantes et que l'on devra se contente d'un résultat quasi-exact, validé par un seuil de tolérance adapté.

Afin d'accélérer cette phase de recherche de la configuration de contact, le retour en arrière se fera par dichotomie. On recule en premier lieu de la moitié du pas de temps puis, si l'inter-pénétration est toujours présente, on recule d'un quart du pas de temps, sinon on avance d'un quart du pas de temps, et ainsi de suite jusqu'à obtenir une précision satisfaisante.

DESSIN

Une fois ce processus achevé, la configuration de contact est retrouvé et les deux sphères ne sont plus en situation de pénétration mutuelle. Il reste à déterminer quelles forces appliquer aux corps pour qu'ils se séparent.

\subsection{Réponse à une collision}

Un coefficient de restitution $0 \leq \epsilon \leq 1$ détermine le taux de rebond d'un corps. Si $\epsilon = 1$, la balle rebondira avec autant d'énergie qu'elle est arrivée (une situation concrètement impossible à retrouver dans le monde réel). Si $\epsilon = 0$, la balle repartira avec une énergie nulle; autrement dit, elle restera collée au plan. Le coefficient de restitution dépend habituellement de la matière que l'on cherche à simuler : une balle en cahoutchouc aura un $\epsilon$ elevé tandis qu'un bloc d'argile aura un $\epsilon$ proche de zéro.

Le coefficient de friction influe sur la réponse rotationnelle qu'une collision est susceptible de provoquer. Si la balle avait été laché, à partir d'une vitesse nulle, sur un plan droit, le rebond aura été purement linéaire. Dans le cas d'un plan incliné, un frottement non perpendiculaire a lieu et la balle va entrer en rotation. Plus le coefficient de friction est élevé et plus cet effet est puissant. On peut simplifier cette notion en disant que le coefficient de friction détermine si un corps est recouvert d'une matière qui accroche ou au contraire qui glisse.

