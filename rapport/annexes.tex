\section{Notions utiles de géométrie}

\subsection{Calcul du point le plus proche}

\subsubsection{Distance entre un point et un segment}

Lors de la simulation, des corps rigides seront amenés à entrer en collision et l'une de tâches de base du moteur physique sera de détecter et de gérer ces évenements. Pour cela, il sera souvent nécessaire de répondre à des question géométriques telles que : un corps touche-t-il avec un autre corps ? Si oui, quels points sont en contact ? Sinon, à quelle distance se trouvent-ils ? Les méthodes que le moteur utilise pour répondre à ce type de question sont décrites dans cette section.

Les techniques géométriques qu ivont être présentées servent deux buts distincts mais en rapport direct : déterminer la distance entre deux objets géométriques et trouver les points les plus proches les séparant. Deux point $x_k$ et $y_l$ appartenant respectivement aux solides $X$ et $Y$ sont considérés comme étant les plus proches si $distance(x_k,y_l) = min(distance(x_i, y_i))$. Autrement dit, ce sont les points de chaque solide qui sont les plus proches l'un de l'autre. On se concentrera sur la recherche des points les plus proches puisque la détermination de la distance entre les deux corps considérés se limite à calculer la distance entre les points les proches.

Premièrement, on veut déterminer le point d'un segment $AB$ le plus proche d'un point $P$ quelconque flottant dans l'espace. Pour commencer, réalisons la même opération mais avec une droite. On cherche donc le point de la droite prolongeant $AB$ le plus proche de $P$. De façon triviale, $C$ est trouvé en projetant $P$ sur la droite $AB$ (c'est l'intersection de $AB$ et de la droite orthogonale à $AB$ passant par $P$.

DESSIN

Le problème du point le plus proche d'un segment est semblable, mis à part qu'un segment ne dispose pas d'une longueur infinie et que la projection de $P$ risque d'en sortir. Si c'est le cas, le point le plus proche $C$ est alors soit $A$ soit $B$.

DESSIN

Tout point $Q$ appartenant au segment $AB$ a pour équation $Q=A+t(B-A)$ avec $0 \leq t \leq 1$. On doit donc calculer le $t$ de la projection du point $P$. S'il est inférieur à $0$, le point le plus proche est $A$. S'il est supérieur à $0$, le point de plus proche est $B$. Sinon le point le plus proche est le point d'équation blabla

\subsubsection{Distance entre un point et un plan}

On veut maintenant calculer le point d'un plan le plus proche d'un point quelconque $P$. On définit un plan par un point lui appartenant et un vecteur lui étant normal.

\subsubsection{Point d'un triangle le plus proche}

On veut déterminer le point d'un triangle $ABC$ le plus proche d'un point quelconque $P$. Comme pour les problèmes précédents, on se basera sur une projection du point $P$, plus précisément sur la projection de $P$ sur le plan auquel appartient $ABC$.

Deux situations sont envisageables, $C$ peut se trouver soit à l'intérieur soit à l'extérieur du triangle $ABC$. S'il est à l'intérieur, le point le plus proche est $C$. S'il se trouve à l'extérieur, alors on est certain que $C$ se trouve sur une des arêtes du triangle. Il nous reste donc à calculer le point de chaque arête le plus proche de $P$ grâce à la méthode segment/point décrite précédemment et à choisir meilleur candidat parmi les trois.

Il n'a pas encore été précisé de méthode à utiliser pour détecter si le point projeté se trouve à l'intérieur de $ABC$. On peut utiliser une des propriétés des triangles qui dicte que la somme des angles reliant les points voisins à un point interne au triangle vaut $2\pi$ radians. 

DESSIN?

\subsubsection{Point d'un polygone le plus proche}

On veut déterminer le point d'un polygone $G$ le plus proche d'un point quelconque $P$. On va une fois encore utiliser une des méthodes précedemment présentées afin de subdiviser le problème. Un polygone est constitué d'un ou de plusieurs triangles, on va donc subdiviser $Q$ en triangles puis calculer le point le plus proche entre $P$ et chaque triangle et conserver le meilleur.

Il a ici été fait le choix de subdiviser le triangle en calculant le centre du polygone puis en reliant chaque sommet consécutifs du polygone à ce centre.

DESSIN

\subsubsection{Point d'une sphère le plus proche}

Le calcule du point d'une sphère le plus proche d'un point quelconque $P$ se fait de façon triviale. Il s'agit de l'intersection du segment reliant le centre de $S$ à $P$ et du rayon de la sphère. Concrètement, on calcule le vecteur séparant le centre de $S$ et $P$, on le normalise puis on le multiplie par le rayon.

DESSIN

\subsubsection{Point de deux segments les plus proches}

megablabla 

\subsection{Distance entre deux corps}

\subsubsection{Différence de Minkowski}

\subsubsection{Algorithme de GJK}

L'algorithme de Gilbert-Johnson-Keerthi, communément appelé GJK, détermine rapidement le membre d'un ensemble de points qui est le plus proche d'un point quelconque. Dans notre cas, ce point quelconque sera l'origine du repère absolu puisque la distance entre le point de plus de la différence de Minkowski et l'origine correspond à la distance entre les corps.

Cet algorithme se base sur l'utilisation de simplex, une structure géométrique contenu dans un corps. Sa dimension peut varier de 1 à la dimension du problème. Le simplex de dimension une correspond à un simple point dans l'espace. Le simplex de dimension deux correpond à un segment. Le simplex de dimension trois correspond à un triangle. Le simplex de dimension quatre correspond à un tétrahèdre.

Le principe de l'algorithme GJK est de partir d'un simplex de base, qui sera ici un des points de la différence de Minkowski, et de le faire évoluer afin qu'il prenne des formes de plus en plus proches de l'origine tant que c'est possible. Une des clés de ce travail est de réduire la dimension du simplex dés que possible, c'est à dire dés que certains de ces caractères deviennent inutiles à la détermination du point le plus proche
