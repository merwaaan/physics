\documentclass{beamer}

\usepackage[francais]{babel}
\usepackage[T1]{fontenc}
\usepackage{amsmath}
\usepackage{color}
\usepackage{tikz}
\usepackage[adobe-utopia]{mathdesign}

\newcommand{\deriv}{\partial \!}

% Couleurs inspirées de famfamfam.com
\definecolor{vert}{HTML}{59FF2D}
\definecolor{bleu}{HTML}{0BCEFF}
\definecolor{rose}{HTML}{FF0B5B}
\definecolor{gris}{gray}{0.5}

\usetikzlibrary{shapes,fit,calc,3d}

\tikzset{line/.style={
    shorten >= -#1,
    shorten <= -#1}}
\tikzset{halfline/.style={
    shorten >= -#1}}
\tikzset{axe/.style={
    ->,
    thin}}
\tikzset{fig/.style={
    thick}}
\tikzset{origine/.style={
    draw=black,
    cross out}}
\tikzset{aabb/.style={
  dashed,
  thick,
  inner sep=0}}
\tikzset{gjknode/.style={
    fill=rose,
    circle}}
\tikzset{gjkedge/.style={
    draw=rose,
    thick}}
\tikzset{gjkdir/.style={
    halfline=0.5cm,
    draw=gris,
    very thick,
    dashed}}
\tikzset{gjkclosest/.style={
    circle,
    fill=bleu}}
\tikzset{rayon/.style={
    ->,
    dashed,
    draw=gris}}

\title{Simulation physique de corps rigides avec interaction}
\author{Merwan Achibet}
\date{}

\begin{document}

\shorthandoff{!} % Pour éviter bug de Tikz + Babel french

\begin{frame}
  \maketitle
\end{frame}

\section{Les moteur physiques}

\begin{frame}
  Moteur physique : système de simulation de comportements mécaniques.

  \begin{itemize}
    \item Industrie, science : précis, modélisation complexe, long
    \item Réalité virtuelle, jeu vidéo : approximatif, temps réel
  \end{itemize}
\end{frame}

\begin{frame}
  Ce projet : un moteur physique basique gérant les interactions entre corps rigides convexes
  \begin{description}
    \item[Convexes] 
    \item[Rigides] Indéformables, incassable
  \end{description}
\end{frame}

\begin{frame}
  Différentes tâches :
  \begin{itemize}
    \item Dynamique.
    \item Gestion des collisions
      \begin{itemize}
        \item Détection
        \item Correction
        \item Réponse
      \end{itemize}
    \end{itemize}
\end{frame}

\section{Dynamique}

\subsection{La composante linéaire}

\begin{frame}
  Entrée : forces environnementales
  Sortie : changement de position

  \begin{align*}
    \vec{a} = \frac{1}{m} \sum_i \vec{F}_i
  \end{align*}

  On connait l'accélération à partir des forces subies.
\end{frame}

\begin{frame}
  Quantités utiles :
  \begin{itemize}
    \item accélération
    \item vitesse
    \item position
  \end{itemize}

  \begin{align*}
    \vec{v} = \frac{\deriv \vec{p}}{\deriv t}
  \end{align*}

  \begin{align*}
    \vec{a} = \frac{\deriv \vec{v}}{\deriv t}
  \end{align*}

  \begin{align*}
    \vec{a} = \frac{\deriv \vec{v}}{\deriv t} = \frac{\partial^2 \vec{p}}{\deriv t}
  \end{align*}
\end{frame}

\begin{frame}
  Intégration d'Euler
  \begin{align*}
    x_{n+1} = x_{n} + x' \deriv t
  \end{align*}

  Appliquée à nos besoins
  \begin{align*}
    \vec{a}_{n+1} &= \frac{1}{m} \sum_i \vec{F}_i \\ \\
    \vec{v}_{n+1} &= \vec{v}_n + \vec{a}_{n+1} \deriv t \\ \\
    \vec{p}_{n+1} &= \vec{p}_n + \vec{v}_{n+1} \deriv t
  \end{align*}

\end{frame}

\begin{frame}
  
  \begin{align*}
    \sum_i \vec{F}_i = \frac{\deriv \vec{L}}{\deriv t} = \frac{\deriv (m\vec{v})}{\deriv t}
  \end{align*}

  \begin{align*}
    \vec{L}_{n+1} &= \vec{L}_n + {\sum_i \vec{F}_i} \\ \\
    \vec{p}_{n+1} &= \vec{p}_n + \frac{1}{m}\vec{L}_{n+1} \deriv t
  \end{align*}

\end{frame}

\subsection{Modélisation d'un corps}

\subsection{La composante angulaire}

\begin{frame}
  Il manque quelquechose

  \'Elan angulaire et orientation sont analogues à l'élan linéaire à la position.

  orientation
\end{frame}


\begin{frame}
  Elan angulaire
\end{frame}

\begin{frame}
  Elan angulaire
\end{frame}

\section{Collisions}

\subsection{Détection}

\begin{frame}  
  Les corps sont testés par paires.

  
\end{frame}

\begin{frame}
  Détection grossière

  \begin{figure}
    \centering
    \begin{tikzpicture}[scale=0.7, transform shape]
  % repère
\draw[->] (-0.2,0) -- (10,0);
\draw[->] (0,-0.2) -- (0,10);
\foreach \i in {0,...,9}
{
  \draw[xshift=\i cm] (0,-1pt) -- (0,1pt);
  \draw[yshift=\i cm] (-1pt,0) -- (1pt,0);
}

  \coordinate (R) at (7,3);
  \node[rectangle,rotate=-30,minimum height=3cm, minimum width=2cm,inner sep=0pt,draw=black] (rect) at (R) {};
  \coordinate (C) at (2,7);
  \node[circle,minimum size=2cm,inner sep=0pt,draw=black] (cerc) at (C) {};

  \begin{scope}[rotate around={-30:(C)}]
    \path (R) node () {}
          ++(-1,-1.5) node (a) {}
          ++(2,0) node (b) {}
          ++(0,3) node (c) {}
          ++(-2,0) node (d) {};
   \end{scope}

  % bounding boxes
  \node[fit=(a) (b) (c) (d),draw=green,dashed] {};
  \node[fit=(cerc),draw=green,dashed] {};

\end{tikzpicture}

    \begin{tikzpicture}[scale=0.7, transform shape]
  % repère
\draw[->] (-0.2,0) -- (10,0);
\draw[->] (0,-0.2) -- (0,10);
\foreach \i in {0,...,9}
{
  \draw[xshift=\i cm] (0,-1pt) -- (0,1pt);
  \draw[yshift=\i cm] (-1pt,0) -- (1pt,0);
}

  \coordinate (R) at (7,3);
  \node[rectangle,rotate=-30,minimum height=3cm, minimum width=2cm,inner sep=0pt,draw=black] (rect) at (R) {};
  \coordinate (C) at (2,7);
  \node[circle,minimum size=2cm,inner sep=0pt,draw=black] (cerc) at (C) {};

  \begin{scope}[rotate around={-30:(C)}]
    \path (R) node () {}
          ++(-1,-1.5) node (a) {}
          ++(2,0) node (b) {}
          ++(0,3) node (c) {}
          ++(-2,0) node (d) {};
   \end{scope}

  % bounding boxes
  \node[fit=(a) (b) (c) (d),draw=green,dashed] {};
  \node[fit=(cerc),draw=green,dashed] {};

\end{tikzpicture}

    \begin{tikzpicture}[scale=0.7, transform shape]
  % repère
\draw[->] (-0.2,0) -- (10,0);
\draw[->] (0,-0.2) -- (0,10);
\foreach \i in {0,...,9}
{
  \draw[xshift=\i cm] (0,-1pt) -- (0,1pt);
  \draw[yshift=\i cm] (-1pt,0) -- (1pt,0);
}

  \coordinate (R) at (7,3);
  \node[rectangle,rotate=-30,minimum height=3cm, minimum width=2cm,inner sep=0pt,draw=black] (rect) at (R) {};
  \coordinate (C) at (2,7);
  \node[circle,minimum size=2cm,inner sep=0pt,draw=black] (cerc) at (C) {};

  \begin{scope}[rotate around={-30:(C)}]
    \path (R) node () {}
          ++(-1,-1.5) node (a) {}
          ++(2,0) node (b) {}
          ++(0,3) node (c) {}
          ++(-2,0) node (d) {};
   \end{scope}

  % bounding boxes
  \node[fit=(a) (b) (c) (d),draw=green,dashed] {};
  \node[fit=(cerc),draw=green,dashed] {};

\end{tikzpicture}

  \end{figure}
\end{frame}

\begin{frame}
  Détection fine

  Somme de Minkoswki : $A \oplus B = \{a + b \mid a \in A, b \in B\}$

  Différence de Minkowski : $A \ominus B = A \oplus (-B)$

  \begin{figure}
    \centering
    \begin{tikzpicture}[scale=0.5,transform shape]
  \begin{scope}[scale=0.7]
    % repère
\draw[->] (-0.2,0) -- (10,0);
\draw[->] (0,-0.2) -- (0,10);
\foreach \i in {0,...,9}
{
  \draw[xshift=\i cm] (0,-1pt) -- (0,1pt);
  \draw[yshift=\i cm] (-1pt,0) -- (1pt,0);
}
  \end{scope}

  % Rectangle
\coordinate (a1) at (4,5);
\coordinate (a2) at (4,3);
\coordinate (a3) at (6,3);
\coordinate (a4) at (6,5);

% Triangle
\coordinate (b1) at (1,2);
\coordinate (b2) at (1,1);
\coordinate (b3) at (3,1);

% Minkowski difference
\coordinate (m1) at (1,4);
\coordinate (m2) at (1,2);
\coordinate (m3) at (3,1);
\coordinate (m4) at (5,1);
\coordinate (m5) at (5,4);


  \draw[black] (a1) -- (a2) -- (a3) -- (a4) -- cycle;
  \node[black] at (5,4) (a) {$A$};

  \draw[black] (b1) -- (b2)-- (b3) -- cycle;
  \node[black] at (1.5,1.5) (b) {$B$};

  \draw[<->,bleu,thick,dashed] (a2) -- ($(b1)!(a2)!(b3)$);
\end{tikzpicture}

    \begin{tikzpicture}[scale=0.55,transform shape]
  \begin{scope}[scale=0.7]
    % repère
\draw[->] (-0.2,0) -- (10,0);
\draw[->] (0,-0.2) -- (0,10);
\foreach \i in {0,...,9}
{
  \draw[xshift=\i cm] (0,-1pt) -- (0,1pt);
  \draw[yshift=\i cm] (-1pt,0) -- (1pt,0);
}
  \end{scope}

  % Rectangle
\coordinate (a1) at (4,5);
\coordinate (a2) at (4,3);
\coordinate (a3) at (6,3);
\coordinate (a4) at (6,5);

% Triangle
\coordinate (b1) at (1,2);
\coordinate (b2) at (1,1);
\coordinate (b3) at (3,1);

% Minkowski difference
\coordinate (m1) at (1,4);
\coordinate (m2) at (1,2);
\coordinate (m3) at (3,1);
\coordinate (m4) at (5,1);
\coordinate (m5) at (5,4);

  
  \draw[black,thick] (m1) -- (m2) -- (m3) -- (m4) -- (m5) -- cycle;
  \node[black] at (3,3) (m) {$A \ominus B$};

  \draw[<->,vert,thick, dashed] (0,0) -- ($(m2)!(0,0)!(m3)$);
\end{tikzpicture}

  \end{figure}

  Particularité : la plus petite distance de la différence de Minkowski à l'origine est la plus petite distance entre les corps $A$ et $B$.
\end{frame}

\begin{frame}
  Comment calculer la plus petite distance de $M$ à l'origine ?

  GJK

  simplex
\end{frame}

\begin{frame}
  fonction de support

  
\end{frame}

\begin{frame}
  \begin{figure}
    \centering
    \begin{tikzpicture}[scale=0.7,transform shape]
	\coordinate (A) at (9,2);
\coordinate (B) at (4,1);
\coordinate (C) at (1,4);
\coordinate (D) at (2,7);
\coordinate (E) at (7,7);
\coordinate (O) at (2,2);

\node[origine] (origin) at (O) {};
\path[draw=black] (A) -- (B) -- (C) -- (D) -- (E) -- cycle;


	\node[draw=red, circle] (a) at (A) {};
  
\end{tikzpicture}

    \begin{tikzpicture}[scale=0.7,transform shape]
	\coordinate (A) at (9,2);
\coordinate (B) at (4,1);
\coordinate (C) at (1,4);
\coordinate (D) at (2,7);
\coordinate (E) at (7,7);
\coordinate (O) at (2,2);

\node[origine] (origin) at (O) {};
\path[draw=black] (A) -- (B) -- (C) -- (D) -- (E) -- cycle;


	\node[gjknode] (a) at (A) {};
	\node[gjknode] (c) at (C) {};
  
  \path[gjkpath] (a) -- (c);
\end{tikzpicture}

    \begin{tikzpicture}[scale=0.7,transform shape]
	\coordinate (A) at (9,2);
\coordinate (B) at (4,1);
\coordinate (C) at (1,4);
\coordinate (D) at (2,7);
\coordinate (E) at (7,7);
\coordinate (O) at (2,2);

\node[origine] (origin) at (O) {};
\path[draw=black] (A) -- (B) -- (C) -- (D) -- (E) -- cycle;


	\node[draw=red, circle] (a) at (A) {};
	\node[draw=red, circle] (c) at (C) {};
	\path[draw=green] (a) -- (c);
  
\end{tikzpicture}

    \begin{tikzpicture}[scale=0.4,transform shape]
	\coordinate (A) at (9,2);
\coordinate (B) at (4,1);
\coordinate (C) at (1,4);
\coordinate (D) at (2,7);
\coordinate (E) at (7,7);
\coordinate (O) at (2,2);

\node[origine] (origin) at (O) {};
\path[draw=black] (A) -- (B) -- (C) -- (D) -- (E) -- cycle;


	\node[gjknode] (a) at (A) {};
	\node[gjknode] (c) at (C) {};
	\node[gjknode] (b) at (B) {};

	\draw[gjkedge] (a) -- (c) -- (b) -- (a);
  \fill[rose,opacity=0.1] (A) -- (C) -- (B);

  \node[gjkclosest] (closest) at ($(C)!(O)!(B)$) {};
\end{tikzpicture}

    \begin{tikzpicture}[scale=0.4,transform shape]
	\coordinate (A) at (9,2);
\coordinate (B) at (4,1);
\coordinate (C) at (1,4);
\coordinate (D) at (2,7);
\coordinate (E) at (7,7);
\coordinate (O) at (2,2);

\node[origine] (origin) at (O) {};
\path[draw=black] (A) -- (B) -- (C) -- (D) -- (E) -- cycle;


	\node[gjknode] (c) at (C) {};
	\node[gjknode] (b) at (B) {};

	\path[gjkedge] (c) -- (b);
  
\end{tikzpicture}

    \begin{tikzpicture}[scale=0.4,transform shape]
	\coordinate (A) at (9,2);
\coordinate (B) at (4,1);
\coordinate (C) at (1,4);
\coordinate (D) at (2,7);
\coordinate (E) at (7,7);
\coordinate (O) at (2,2);

\node[origine] (origin) at (O) {};
\path[draw=black] (A) -- (B) -- (C) -- (D) -- (E) -- cycle;


  \node[gjkclosest] (closest) at ($(C)!(O)!(B)$) {};

  \node at ($(closest)+(0.4,0)$) (lclosest) {$S_2$};
\end{tikzpicture}

  \end{figure}
\end{frame}

\subsection{Correction}

\begin{frame}
  Simulation discrète : pas de temps fixe

  Les collisions sont toujours pénétrantes

  \begin{figure}
    \centering
    \begin{tikzpicture}[scale=0.7, transform shape]

\coordinate (A1) at (0,0);
\coordinate (A2) at (6.7,0);
\coordinate (B) at (8,0);

\draw node[fig,
  circle,
  minimum size=2cm,
  draw=bleu,
  dotted] (a1) at (A1) {};

\node[fig,
  circle,
  minimum size=2cm,
  draw=bleu] (a2) at (A2) {};

\node[fig,
  circle,
  minimum size=2cm,
  draw=rose] (b) at (B) {};

\draw[->,thick,vert] (A1) to node[black,auto] {$+ \deriv t$} (A2);

\end{tikzpicture}

  \end{figure}
\end{frame}

\begin{frame}
  L'intégration peut aussi se faire en arrière !

  On procède par dichotomie.

  \begin{figure}
    \centering
    \begin{tikzpicture}[scale=0.7, transform shape]

\coordinate (A1) at (0,0);
\coordinate (B) at (8,0);

\coordinate (A2) at (6.7,0);
\coordinate (A2prec) at (A1);
\draw node[fig,
  circle,
  minimum size=2cm,
  draw=blue,
  dotted] (a1) at (A1) {};

\node[fig,
  circle,
  minimum size=2cm,
  draw=blue] (a2) at (A2) {};

\node[fig,
  circle,
  minimum size=2cm,
  draw=rose] (b) at (B) {};

\node[
  circle,
  dashed,
  minimum size=2.5cm,
  draw=rose] (btol) at (B) {};

\draw[->,thick,vert,font=\Large] (A2prec) to node[black,auto] {$+ \deriv t$} (A2);

\coordinate (A1) at ($(A1) + (0,-3)$);
\coordinate (A2) at ($(A2) + (0,-3)$);
\coordinate (A2prec) at ($(A2prec) + (0,-3)$);
\coordinate (B) at ($(B) + (0,-3)$);
\coordinate (A2precT) at (A2prec);
\coordinate (A2prec) at (A2);
\coordinate (A2) at ($(A2)!0.5!(A2precT)$);
\draw node[fig,
  circle,
  minimum size=2cm,
  draw=blue,
  dotted] (a1) at (A1) {};

\node[fig,
  circle,
  minimum size=2cm,
  draw=blue] (a2) at (A2) {};

\node[fig,
  circle,
  minimum size=2cm,
  draw=rose] (b) at (B) {};

\node[
  circle,
  dashed,
  minimum size=2.5cm,
  draw=rose] (btol) at (B) {};

\draw[->,thick,vert,font=\Large] (A2prec) to node[black,auto,swap] {$- \frac{\deriv t}{2}$} (A2);

\coordinate (A1) at ($(A1) + (0,-3)$);
\coordinate (A2) at ($(A2) + (0,-3)$);
\coordinate (A2prec) at ($(A2prec) + (0,-3)$);
\coordinate (B) at ($(B) + (0,-3)$);
\coordinate (A2precT) at (A2prec);
\coordinate (A2prec) at (A2);
\coordinate (A2) at ($(A2)!0.5!(A2precT)$);
\draw node[fig,
  circle,
  minimum size=2cm,
  draw=blue,
  dotted] (a1) at (A1) {};

\node[fig,
  circle,
  minimum size=2cm,
  draw=blue] (a2) at (A2) {};

\node[fig,
  circle,
  minimum size=2cm,
  draw=rose] (b) at (B) {};

\node[
  circle,
  dashed,
  minimum size=2.5cm,
  draw=rose] (btol) at (B) {};

\draw[->,thick,vert,font=\Large] (A2prec) to node[black,auto] {$+ \frac{\deriv t}{4}$} (A2);

\coordinate (A1) at ($(A1) + (0,-3)$);
\coordinate (A2) at ($(A2) + (0,-3)$);
\coordinate (A2prec) at ($(A2prec) + (0,-3)$);
\coordinate (B) at ($(B) + (0,-3)$);
\coordinate (A2precT) at (A2prec);
\coordinate (A2prec) at (A2);
\coordinate (A2) at ($(A2)!-1!(A2precT)$);
\draw node[fig,
  circle,
  minimum size=2cm,
  draw=blue,
  dotted] (a1) at (A1) {};

\node[fig,
  circle,
  minimum size=2cm,
  draw=blue] (a2) at (A2) {};

\node[fig,
  circle,
  minimum size=2cm,
  draw=rose] (b) at (B) {};

\node[
  circle,
  dashed,
  minimum size=2.5cm,
  draw=rose] (btol) at (B) {};

\draw[->,thick,vert,font=\Large] (A2prec) to node[black,auto] {$+ \frac{\deriv t}{4}$} (A2);

\coordinate (A1) at ($(A1) + (0,-3)$);
\coordinate (A2) at ($(A2) + (0,-3)$);
\coordinate (A2prec) at ($(A2prec) + (0,-3)$);
\coordinate (B) at ($(B) + (0,-3)$);
\coordinate (A2precT) at (A2prec);
\coordinate (A2prec) at (A2);
\coordinate (A2) at ($(A2)!0.5!(A2precT)$);
\draw node[fig,
  circle,
  minimum size=2cm,
  draw=blue,
  dotted] (a1) at (A1) {};

\node[fig,
  circle,
  minimum size=2cm,
  draw=blue] (a2) at (A2) {};

\node[fig,
  circle,
  minimum size=2cm,
  draw=rose] (b) at (B) {};

\node[
  circle,
  dashed,
  minimum size=2.5cm,
  draw=rose] (btol) at (B) {};

\draw[->,thick,vert,font=\Large] (A2prec) to node[black,auto,swap] {$- \frac{\deriv t}{8}$} (A2);

\end{tikzpicture}

  \end{figure}
\end{frame}

\subsection{Réponse}

\begin{frame}
  Un corps
  \begin{itemize}
    \item sommets
    \item arêtes
    \item faces
  \end{itemize}

  On s'intéresse uniquement aux contacts sommet-face et arête-arête.

  \begin{figure}
    \centering
    \begin{tikzpicture}[scale=0.4]
  % Avant.
  \begin{scope}[canvas is xy plane at z=4]
    \path[fill=gris!20] (0,0) rectangle (4,4);
  \end{scope}
  % Droite.
  \begin{scope}[canvas is zy plane at x=4]
    \path[fill=gris!50] (0,0) rectangle (4,4);
  \end{scope}
  % Haut.
  \begin{scope}[canvas is zx plane at y=4]
    \path[fill=gris!70] (0,0) rectangle (4,4);
  \end{scope}

  % Avant.
  \begin{scope}[canvas is xy plane at z=2]
    \path[fill=gris!20] (2,4) rectangle (6,8);
  \end{scope}
  % Droite.
  \begin{scope}[canvas is yz plane at x=6]
    \path[fill=gris!50] (4,2) rectangle (8,-2);
  \end{scope}
  % Haut.
  \begin{scope}[canvas is xz plane at y=8]
    \path[fill=gris!70] (2,2) rectangle (6,-2);
  \end{scope}
\end{tikzpicture}

    w\begin{tikzpicture}[scale=0.4]
  % Avant.
  \begin{scope}[canvas is xy plane at z=4]
    \path[draw=black, thick] (0,0) rectangle (4,4);
  \end{scope}
  % Droite.
  \begin{scope}[canvas is zy plane at x=4]
    \path[draw=black, thick] (0,0) rectangle (4,4);
  \end{scope}
  % Arrière.
  \begin{scope}[canvas is xy plane at z=0]
    \path[draw=black,dashed,thin] (0,0) rectangle (4,4);
  \end{scope}
  % Gauche.
  \begin{scope}[canvas is yz plane at x=0]
    \path[draw=black,dashed,thin] (0,0) rectangle (4,4);
  \end{scope}
  % Haut.
  \begin{scope}[canvas is zx plane at y=4]
    \path[draw=black, thick] (0,0) rectangle (4,4);
  \end{scope}

  % Avant.
  \begin{scope}[canvas is xy plane at z=2]
    \path[draw=black, thick] (2,4) rectangle (6,8);
  \end{scope}
  % Droite.
  \begin{scope}[canvas is yz plane at x=6]
    \path[draw=black, thick] (4,2) rectangle (8,-2);
  \end{scope}
  % Haut.
  \begin{scope}[canvas is xz plane at y=8]
    \path[draw=black, thick] (2,2) rectangle (6,-2);
  \end{scope}
  % Arrière.
  \begin{scope}[canvas is xy plane at z=-2]
    \path[draw=black,dashed,thin] (2,4) rectangle (6,8);
  \end{scope}
  % Gauche.
  \begin{scope}[canvas is yz plane at x=2]
    \path[draw=black,dashed,thin] (4,2) rectangle (8,-2);
  \end{scope}

  \begin{scope}[canvas is xz plane at y=4]
    \node[fill=bleu,circle] at (2,0) (c1) {};
    \node[fill=rose,circle] at (2,2) (c2) {};
    \node[fill=bleu,circle] at (4,2) (c3) {};    
    \node[fill=rose,circle] at (4,0) (c4) {};
  \end{scope}
\end{tikzpicture}

  \end{figure}
\end{frame}

\begin{frame}
  Un \textit{contact}
  \begin{itemize}
    \item position
    \item normale
    \item temps
  \end{itemize}

  Un contact = une impulsion

  \begin{align*}
    J = \vec{n} 
    \frac{-(1 + \varepsilon) v_r}{
      \frac{1}{m_A} +
      \frac{1}{m_B} +
      \vec{n}
      (I_A^{-1} (\vec{r}_A \times \vec{n})) \times \vec{r}_A +
      (I_B^{-1} (\vec{r}_b \times \vec{n})) \times \vec{r}_B
    }
  \end{align*}

  r
\end{frame}

\begin{frame}

\end{frame}

\section{Le moteur}

\subsection{Organisation}

\begin{frame}
  \begin{enumerate}
    \item détection de collision
    \item Application de forces environnementales
    \item intégration
  \end{enumerate}

  défauts
\end{frame}

\subsection{Démonstrations}

\begin{frame}
  \begin{enumerate}
    \item détection de collision
    \item Application de forces environnementales
    \item intégration
  \end{enumerate}

  mieux
\end{frame}

\subsection{Perspectives}

\begin{frame}
  \begin{figure}
    \centering
    \begin{tikzpicture}[scale=0.7, transform shape]
  \coordinate (AT1) at (0,0);
\coordinate (AT2) at (8,0);
\coordinate (B) at (6,0);

\draw node[fig,
  circle,
  minimum size=2cm,
  draw=bleu,
  dotted] (a1) at (AT1) {};

\node[fig,
  circle,
  minimum size=2cm,
  draw=bleu] (a2) at (AT2) {};
\node[fig,
  circle,
  minimum size=1cm,
  draw=rose] (b) at (B) {};



  \draw[->,thick,vert] (AT1) to node[black,auto] {$\deriv t$} (AT2);
\end{tikzpicture}

    \begin{tikzpicture}[scale=0.7, transform shape]
  \coordinate (AT1) at (0,0);
\coordinate (AT2) at (8,0);
\coordinate (B) at (6,0);

\draw node[fig,
  circle,
  minimum size=2cm,
  draw=bleu,
  dotted] (a1) at (AT1) {};

\node[fig,
  circle,
  minimum size=2cm,
  draw=bleu] (a2) at (AT2) {};
\node[fig,
  circle,
  minimum size=1cm,
  draw=rose] (b) at (B) {};



  \path[rayon] (a1.north) -- (a2.north);
  \path[rayon] (a1.west) -- (a2.west);
  \path[rayon] (a1.north west) -- (a2.north west);
  \path[rayon] (a1.south) -- (a2.south);
  \path[rayon] (a1.south west) -- (a2.south west);
\end{tikzpicture}

    \begin{tikzpicture}[scale=0.7, transform shape]
  \coordinate (AT1) at (0,0);
\coordinate (AT2) at (8,0);
\coordinate (B) at (6,0);

\draw node[fig,
  circle,
  minimum size=2cm,
  draw=bleu,
  dotted] (a1) at (AT1) {};

\node[fig,
  circle,
  minimum size=2cm,
  draw=bleu] (a2) at (AT2) {};
\node[fig,
  circle,
  minimum size=1cm,
  draw=rose] (b) at (B) {};



  \coordinate (a1n) at ($(a1.north)+(0,0.2)$);
  \coordinate (a2s) at ($(a2.south)+(0,-0.2)$);

  \draw[gris] (a1n) arc (90:270:1.2cm) --
  (a2s) arc (-90:90:1.2cm) --
  (a1n);
\end{tikzpicture}

  \end{figure}
\end{frame}

\begin{frame}
  
\end{frame}


\end{document}
