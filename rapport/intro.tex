\section{Introduction}

\subsection{Applications}

Dans de nombreux domaines liés à l'informatique, la reproduction réaliste de phénomènes physiques du monde réel est recherchée. blabla

Les simulations physiques sont présentes dans de nombreux systèmes complexes. Parfois mises sur le devant de la scène, au c\oe ur d'applications dont l'utilité directe est de les étudier, ou parfois tenant un rôle de fond plus discret mais néanmoins nécessaire à l'atteinte d'un certain réalisme. Certaines simulations industrielles se doivent par exemple de reproduire fidèlement des situations réelles et sont supposées pouvoir prédire de manière précise l'évolution d'un environnement. Dans ce cadre, on souhaite obtenir des résultats réalistes. Dans d'autres contextes, comme la réalité virtuelle ou le jeu vidéo, l'obtention d'une simulation comparable à celle du monde physique est attendue mais les enjeux étant moins importants, certains écarts peuvent être acceptés. On préferera ici bénéficier d'une vitesse d'éxécution confortable que de résultats absolument exactes.

Le champ de la simulation physique est vaste et de nombreuses approches sont reconnues. Le choix quant à la direction à prendre dans la conception d'un moteur physique dépend naturellement du type de simulation que l'on espère produire. Les simulations précises ne s'embarassent pasde la contrainte de la vitesse et peuvent se permettre d'employer des techniques d'analyse linéaire complexes. La détermination de l'évolution d'un environnement peuplé par des corps interagissant physiquement entre eux est alors réduite à la résolution de système linéaires. Ces systèmes peuvent se réveler extrêmement complexes et forcent à l'utilisation d'outils mathématiques de pointes, l'analyse quadratique par exemple.

Par opposition aux simulations précises, on trouve les simulations temps-réel, dont la contrainte principale est l'obtention de résultats acceptables en respectant une certaine limite de temps. L'une des branches de l'informatique utilisant ce type de moteurs physique est celle de l'informatique de loisir, notamment le jeu vidéo. Les mondes virtuels créés dans le cadre de ces productions étant de plus en proche de la réalité, que ce soit au niveau des graphismes ou de la gestion du son, qu'il est attendu que la partie physique soit assurée avec le même panache. blabla

\subsection{Travail à accomplir}

Le but de ce projet est de concevoir un moteur physique de base permettant de reproduire de la manière la plus réaliste possible les comportements d'objets évoluant dans l'espace et soumis à différents types de forces, telles que la gravité ou la friction. L'appelation de moteur physique est large, il nous faut avant préciser l'étendue du travail à accomplir.

La contrainte principale sera le temps. On veut concevoir une simulation pouvant être éxecuté en temps réel de telle façon que si l'utilisateur modifie l'environnement de la simulation à un instant quelconque, une réaction à cette interaction soit immédiatement générée. Pour cette raison, certains raccourcis dans les calculs ainsi que plusieurs approximations seront tolérées. Le fonctionnement du moteur se base néanmoins sur des formules de la physique dite de Newton et ses résultats ne devront pas trop s'éloigner de ceux que l'on retrouverait dans la réalité. On s'impose le limiter nos simulations aux corps rigides et convexes. Le fait qu'un corps soit rigide, ou indéformable, dénote l'impossibilité d'altérer sa structure. Dans notre simulation, si un objet de cinq tonnes est posé sur une minuscule objet de trois grammes, il ne sera ni aplati, ni endommagé. On choisit aussi de se limiter aux objets convexes car il est plus aisé de gérer leurs interactions. On proposera néanmoins une piste pour étendre notre simulations aux corps concaves dans la dernière partie.

La simulation devra reproduire tous les phénomènes réels que l'on veut imiter mais si dans le monde physique toute la progression menant à la chute d'un objet soumis à la gravité semble naturellement continue, plusieurs phases devront être traversées avant chaque pas de la simulation. En premier lieu, il faudra faire évoluer les quantités entrant en jeu dans l'action dynamiques d'un objet, on pense par exemple au changement de position en fonction d'une certaine vitesse ou d'une accélération.  blabla

Ensuite, on greffera au moteur une gestion de l'aspect kinétique, c'est à dire du comportement d'objets indépendants et soumis à des forces externes. blabla

Une fois que nos solides disposeront d'un comportement réaliste en tant qu'entités, nous les ferons réagir entre eux. Concrètement cette partie sera divisée en deux sections. Tout d'abord, on veut déterminer si deux objets sont en contact, puis l'on va provoquer des réactions collisionnelles proportionnelles à leur vitesse relative.
