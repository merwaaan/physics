\documentclass{beamer}

\usepackage[francais]{babel}
\usepackage[utf8]{inputenc}
\usepackage{tikz}

\title{TITRE}
\author{Merwan Achibet}
\date{}

\begin{document}

\begin{frame}
  \maketitle
\end{frame}

\section{Introduction}

\begin{frame}
  Qu'est-ce qu'un moteur physique ?

  Exemples
\end{frame}

\begin{frame}
  Différentes tâches :
  \begin{itemize}
    \item Simulation des forces.
    \item Détection des collisions
    \item Réponse aux collisions.
    \end{itemize}
\end{frame}

\begin{frame}
  Ce projet :
  \begin{itemize}
    \item Moteur physique de base
    \item Corps convexes
    \item Corps indéformables
  \end{itemize}
\end{frame}

\section{Dynamique}

\begin{frame}
  Lois de base
  équation

  Moteur physique :
  entrée : forces environnementales
  sortie : changement de position

  relation de dérivation

  Dans l'autre sens ?
\end{frame}

\begin{frame}
  On approxime l'intégration

  Méthode d'Euler générale

  Méthode d'Euler appliquée au moteur
\end{frame}

\begin{frame}
  Pour le mouvement angulaire ?

  Même principe avec plus de variables auxiliaires :

  tenseur d'inertie, etc
\end{frame}

\section{Collisions}

\begin{frame}
  Objet testés deux à deux

  On présentera une méthode efficace mais coûteuse si beaucoup exécutée.

  Détection grossière puis détection fine
\end{frame}

\begin{frame}
  Détection grossière

  AABB : construction

  AABB : comparaison
\end{frame}

\begin{frame}
  Détection fine

  Somme
\end{frame}

\begin{frame}
  Différence de Minkowski
\end{frame}

\begin{frame}
  GJK
\end{frame}

\begin{frame}
  GJK
\end{frame}

\begin{frame}
  Correction dichotomique
\end{frame}

\begin{frame}
  Types de contact

  Vertex/face

  Edge/Edge
\end{frame}

\begin{frame}
  Impulsion
\end{frame}

\section{Le moteur}

\subsection{Organisation}

\begin{frame}
  Algo
\end{frame}

\subsection{Démonstrations}

\begin{frame}
  Algo++
\end{frame}

\subsection{Perspectives}

\begin{frame}
  Tunneling
\end{frame}

\begin{frame}
  Partitionnement de l'espace (?)
\end{frame}

\end{document}
