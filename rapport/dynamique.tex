\section{Dynamique} 

\subsection{La composante linéaire du mouvement}

Nous allons pour l'instant nous concentrer sur l'aspect cinématique
d'un corps rigide, c'est à dire son mouvement lorsqu'il n'est soumis à
aucune force extérieure. Dans un premier temps, seule la composante
linéaire du mouvement sera étudiée et les objets dont nous allons
simuler le comportement seront réduits à de simples particules. Les
figures présentées tout au long de ce rapport sont en deux dimensions
pour des raisons de clarté mais le principe reste similaire lorsqu'il
est étendu à la troisième dimension.

La quantité physique la plus perceptible visuellement pour un
spectateur est la position $\vec p$ d'un corps. Pour notre affichage
final, c'est cette valeur à tout instant $t$ de la simulation que l'on
veut déterminer. Le travail de la partie dynamique du moteur physique
revient donc à déterminer la nouvelle position d'un objet à partir de
la connaissance des forces qui lui sont appliquées. Un corps possède
aussi une vitesse $\vec v$, qui correspond à la variation de sa
position pour une unité de temps. On note cette relation sous la forme
dérivée :

\begin{align*}
  \vec{v} = \frac{\deriv \vec{p}}{\deriv t}
\end{align*}

Pareillement, l'accéleration $\vec a$, qui apparaissaît plus tôt dans
la seconde loi de Newton, correspond à la variation de la vitesse par
rapport à une unité de temps.

\begin{align*}
  \vec{a} = \frac{\deriv \vec{v}}{\deriv t}
\end{align*}

Par transitivité, on confirme que la position est dépendante de
l'accélération.

\begin{align*}
  \vec{a} = \frac{\deriv \vec{v}}{\deriv t} = \frac{\deriv^2
    \vec{p}}{\deriv t}
\end{align*}

Grâce à cette équation, on peut calculer l'accélération d'un objet à
partir de sa variation de position. Dans le moteur physique, ce sera
en fait l'opération inverse qui devra être effectuée : on connaîtra
les forces appliquées, on en déduira par la seconde loi de Newton
l'accélération induite puis on calculera le changement de position. Il
faut donc exploiter le fait que ces trois quantités partagent aussi
des relations de primitive.

\begin{align*}
  \vec{p} = \int \vec{v}\; \deriv t = \int \vec{a}\; \deriv t^2
\end{align*}

Récapitulons. Chacun des corps rigides dont l'on veut simuler
l'évolution possèdera trois quantités sous forme vectorielle :
position $\vec p$, vitesse $\vec v$ et accélération $\vec a$. L'une
des tâches de base du moteur physique sera de traduire l'application
de forces sur un corps en un changement de sa position. On sait que
les quantités énoncées entretiennent des relations de dérivation, il
faudra donc procéder par intégration pour calculer la nouvelle
position d'un objet à partir de son accélération.

\subsection{Intégration}

Les trois quantités physiques entrant en jeu dans les mouvements
linéaires ont été présentées. \'Etudions maintenant de façon plus
concrète sur quels calculs se basera l'exercice le plus élémentaire du
moteur physique.

Les phénomènes mécaniques du monde réel évoluent de façon continue
mais notre simulation ne peut pas s'autoriser ce luxe. Le modèle du
moteur physique sera donc discret et avancera par pas de temps fixe
$\deriv t$. \`A chaque mise à jour de la simulation, des intégrations
doivent être réalisées pour déterminer le changement d'état d'un corps
d'un instant $t_n$ à un instant $t_{n+1} = t_n + \deriv t$. Toujours
pour des raisons d'efficacité, nous ne pouvons pas nous permettre
d'allouer un temps de calcul trop important à cette phase de la mise à
jour et nous devons trouver un moyen d'approximer ces intégrales.

Parmi les techniques classiques d'intégration approximative, on trouve
l'intégration d'Euler. Cette méthode part du principe que l'on dispose
de la valeur initiale $x_0$ de la quantité que l'on souhaite faire
évoluer ainsi que de son taux de changement $x'$ et de la variation de
temps par rapport à l'état précédent.

\begin{align*}
  x_{n+1} = x_{n} + x' \deriv t
\end{align*}

Si l'on adapte cette méthode à notre problème, on obtient la
succession de calculs suivante :

\begin{align*}
  \vec{a}_{n+1} &= \frac{\sum \vec{F}_i}{m} \\ \vec{v}_{n+1} &=
  \vec{a}_n + \vec{a}_{n+1} \deriv t \\ \vec{p}_{n+1} &= \vec{p}_n +
  \vec{v}_{n+1} \deriv t
\end{align*}

On calcule l'accélération à un instant $t$ puis on intègre en fonction
du temps jusqu'à obtenir la nouvelle position du corps. Ce processus
doit être repété à chaque mise à jour du système et ce, pour chaque
corps. On aurait pu choisir une autre méthode d'intégration, telle que
l'intégration Runge-Kutta d'ordre 4 qui divise un pas de temps en
plusieurs segments [source], ou bien telle que l'intégration de Verlet
qui dispose d'une meilleure stabilité [source], mais Euler reste le
choix le plus économique et fournit des résultats relativement
corrects.

Afin de réduire la complexité de la structure informatique qui
représentera un corps rigide, nous allons introduire la notion d'élan
linéaire. Pour un corps rigide, l'élan linéaire $\vec{L}$ est le
produit de la masse et de la vitesse. Cette nouvelle quantité a pour
avantage majeur de posséder comme primitive la variation de force
exercée sur le corps.

\begin{align*}
  {\sum \vec{F}_i} = \frac{\deriv \vec{L}}{\deriv t} = \frac{\deriv (m\vec{v})}{\deriv t}
\end{align*}

Ce qui signifie que l'on peut réduire l'intégration de l'état d'un
corps à :

\begin{align*}
  \vec{L}_{n+1} &= \vec{L}_n + {\sum \vec{F}_i} \\ \vec{p}_{n+1} &=
  \vec{p}_n + \frac{\vec{L}_{n+1}}{m} \deriv t
\end{align*}

L'élan linéaire remplace l'accélération dans la définition d'un corps
et permet de calculer sa vitesse si besoin en est. Cette organisation
permet à première vue une intégration plus courte et une occupation
moindre de la mémoire mais les avantages majeurs de ce choix prendront
tous leurs sens lorsque le mouvement angulaire sera introduit.

\subsection{Modélisation d'un corps rigide}

Nous avons décrit dans la partie précédente les quantités régissant le
mouvement linéaire d'une particule ainsi que la façon dont elles
évoluent mais le moteur physique que l'on conçoit a pour visée de
simuler les interactions entre des corps rigides à volume
convexe. Comment peut-on étendre les principes énoncés pour des
particules à ce modèle ?

On pourrait en premier lieu penser à représenter un tel corps par une
liste de particules, chacune placée à un coin de l'objet. Chaque
particule évoluerait indépendement et des contraintes de cohésion
entre particules voisines seraient appliquées pour empêcher toute
déformation du corps. Cette méthode est envisageable mais elle
présente plusieurs désavantages. Premièrement, les règles de cohésion
à mettre en place nécessiteraient des traitements supplémentaires, et
donc un temps de calcul plus élevé. Deuxièment, un corps devrait
passer par autant d'intégrations qu'il a de points à chaque mise à
jour. Il existe une solution plus simple et plus élégante qui permet
de réduire les mouvements linéaires d'un corps rigide à ceux d'une
unique particule judicieusement placée.

Introduisons en premier lieu la notion de repères absolu et local. Le
repère absolu est le référentiel orthogonal dont l'origine sert de
centre à l'environnement de la simulation. Un repère local est un
référentiel qui est unique à chaque corps et dont le centre se situe à
l'intérieur même de cet objet.

\begin{figure}
  \centering
  \begin{tikzpicture}
  % repère
\draw[axe] (-0.2,0) -- (10,0);
\draw[axe] (0,-0.2) -- (0,10);


  % figure
  \coordinate (C) at ++(6,5);
  \draw (C) circle (2);

  % repère local  
  \begin{scope}[rotate around={-30:(C)}]
    \draw[->] (C) -- +(4,0);
    \draw[->] (C) -- +(0,4);
  \end{scope}
\end{tikzpicture}

  \caption{Les repères absolu et local d'un rectangle}
  \label{reperelocal}
\end{figure}

La position exacte de l'origine du repère local dépend de la position
des points qui forment l'objet mais il ne s'agit pas d'un simple
centre géométrique puisque la masse de chaque point est aussi un
facteur déterminant. Cette position se nomme le centre de masse, ou
barycentre, et sera calculée par la formule suivante, o\`u $M$ est la
masse totale des points du corps, et $m_i$ et $\vec{p}_i$
respectivement la masse et la position du point $i$ dans le repère
absolu.

\begin{align*}
  C = \frac{1}{M} \sum m_i \vec{p}_i
\end{align*}

Une fois le centre de masse défini, la position locale $\vec{r}_i$
d'une particule $i$ peut être calculée en fonction de sa position
absolue $\vec{p}_i$ par :

\[\vec{r}_i = \vec{p}_i - C\]

Grâce au centre de masse, on a une seule position à faire évoluer,
quelle que soit la complexité de la structure du corps concerné.

\subsection{La composante angulaire du mouvement}

Le modèle que nous avons défini est encore incomplet puisqu'il ne
prend pas en compte la composante rotationnelle des mouvements dont
l'on peut être témoin dans un environnement réel. Les particules étant
de simples points flottants dans l'espace, cela ne posait pas de
problème précédemment mais le moteur physique que l'on conçoit doit
gérer des volumes plus complexes. Imaginons une boîte cubique que l'on
lancerait en l'air, si aucune rotation n'apparaît (si la base de la
boîte reste parallèle au sol), l'imitation du réel que l'on souhaite
reproduire perd toute crédibilité.

Les quantités physiques entrant en jeu dans la décomposition d'un
déplacement angulaire sont analogues à celles présentées dans la
partie traitant de la dynamique linéaire : à la position et à l'élan
linéaire correspondent l'orientation et l'élan angulaire.

\subsubsection{L'orientation}

De la même façon que la position représentait visuellement l'état d'un
corps au sein de la composante linéaire, un corps doit posséder une
orientation. En deux dimensions, un nombre flottant représentant
l'angle du corps par rapport à un axe fixe suffirait à décrire
l'orientation d'un objet mais pas dans notre environnement en trois
dimensions.

Le repère local d'un corps a été introduit dans la partie précédente
et se résumait à un centre de masse faisant office d'origine, mais un
repère possède aussi des axes et ceux du repère local ne sont pas
nécessairement alignés avec ceux du repère absolu. Pour représenter la
direction des axes du repère local, et donc l'orientation du corps à
qui il appartient, on utilisera une matrice de dimension 3 dans
laquelle chaque vecteur colonne correspondra à un des axes du repère
local \cite{witkit97}. Pour illustrer ces propos, analysons la matrice
d'orientation égale à la matrice identité d'ordre 3.

\[
\begin{pmatrix}
  1 & 0 & 0 \\ 0 & 1 & 0 \\ 0 & 0 & 1
\end{pmatrix}
\rightarrow
\begin{pmatrix}
  1 \\ 0 \\ 0
\end{pmatrix}
\begin{pmatrix}
  0 \\ 1 \\ 0
\end{pmatrix}
\begin{pmatrix}
  0 \\ 0 \\ 1
\end{pmatrix}
\]

Si l'on isole les vecteurs colonnes de cette matrice identité, on
remarque que chacun correspond à un des axes du repère global. Le
premier vecteur colonne d'une matrice d'orientation correspond donc à
l'axe $x$ du repère local. Il en est de même pour la seconde colonne
avec l'axe $y$ et la troisième colonne avec l'axe $z$. [détails]

\subsubsection{\'Elan angulaire}

On pourrait utiliser vitesse et accélération angulaires en tant que
variables d'état, mais comme pour la dynamique linéaire on choisit de
remplacer ces deux valeurs par un unique élan angulaire. L'élan
angulaire $\vec A$ possède comme primitive la variation de force
exercée sur un corps. Mais attention, sa définition diffère de l'élan
linéaire dans la mesure o\`u il n'est en relation qu'avec la
composante angulaire d'une force.

En effet, il est primordial de faire la distinction entre l'influence
linéaire et l'influence angulaire qu'une force exerce sur un
corps. Quel que soit le point d'un objet sur lequel une force est
exercée, la quantité d'élan linéaire ajoutée est la même. Par contre,
la quantité d'élan angulaire transmis par une force dépend de son
point d'application; plus précisément de son excentricité par rapport
au centre de masse. Imaginons une boîte cubique flottant en état
d'apesanteur. Si l'on exerce une légère poussée sur le milieu d'une de
ses faces alors la boîte subira une translation. Si l'on applique
maintenant une pression toujours dans la même direction mais cette
fois sur l'un des coins de la boîte, la même translation sera
accompagné d'une rotation autour du centre de masse. On quantifie
cette effet rotatif par le couple $\tau$, qui dépend de la position
$\vec C$ du centre de masse et de la position $\vec x$ du point
d'application de la force.

\begin{align*}
  \tau = (\vec{x} - \vec{C}) \times \vec{F}
\end{align*}

\subsection{Quantités auxiliaires}

On sait désormais qu'un corps possède une orientation et un élan
angulaire, on sait aussi comment passer de l'application d'une force à
une variation d'élan angulaire. Néanmoins, l'intégration de la
composante rotative d'un mouvement n'est pas aussi directe que pour sa
version linéaire. Nous avons encore besoin de faire appel à plusieurs
quantités auxiliaires, telles que le tenseur d'inertie et le tenseur
d'inertie instantanée.

Concrètement, un tenseur d'inertie est une matrice de dimension 3 dont
les coefficients servent de facteurs lors du calcul de l'orientation à
partir de l'élan angulaire. Il dépend directement de la forme du corps
considérés et affecte les axes de rotation principaux de ce
dernier. Pour le calculer on doit effectuer des calculs d'intégration
par rapport au volume du corps considéré. Les tenseurs d'inertie des
volumes usuels sont souvent donnés dans les manuel de physique.

Le tenseur d'inertie instantanée est le pendant local du torseur
d'inertie. blabla

\begin{align*}
  I(t) = R(t) I R(t)^T
\end{align*}

\subsection{Intégration complète}

\begin{align*}
  I(t) = R(t) I R(t)^T \\ \vec{w}_{n+1} = \frac{\vec{A}}{I(t)} \\ R =
  \vec{w} R \deriv t
\end{align*}

\subsubsection{Correction de l'intégration}

Notre processus d'intégration est désormais complet et nous pouvons
simuler les réactions de corps soumis à des forces externes. Il reste
pourtant un problème à résoudre : à cause de l'utilisation de nombres
à virgule flottante, de légères erreurs de précision apparaîssent à
chaque intégration du système et s'accumulent. L'intégration linéaire
ne passe que par des opérations de base et peu d'erreur se font
ressentir. Par contre, l'erreur la plus flagrante touche l'orientation
des objets. Le problème principal vient du fait que la matrice
d'orientation $R$, qui détermine la direction des axes du repère
local, perd progressivement de son orthonormalité : les axes ne sont
plus orthogonaux entre eux et leur longueur n'est plus égale à 1. Lors
du rendu graphique de la simulation, on utilise la liste des points
qui forment un corps ainsi que sa matrice de rotation pour le
dessiner, et à cause de cette erreur de précision les solides dessinés
sont déformés.

Afin de contrer cet effet indésirable, on ajoutera deux étapes
supplémentaires à la fin de chaque intégration. Premièrement, il nous
faudra normaliser les axes du repère local. Chaque axe correspondand à
un vecteur colonne de la matrice d'orientation, il faudra donc les
extraire un par un et recalculer leur magnitude pour s'assurer de leur
normalité. En second lieu, il sera nécessaire de réorthogonaliser le
repère local, autrement dit s'assurer que ses axes restent orthognaux
entre eux. Pour cela, on orthogonalisera la matrice de rotation par
l'algorithme de Gram-Schmidt. [détails]
